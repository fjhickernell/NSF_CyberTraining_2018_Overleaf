\section*{Biographical Sketch for David Minh}

\begin{enumerate}[(a)]

\item Professional Preparation
\begin{itemize}
\item University of California, Berkeley, CA. Chemistry. B.A. 2003
\item University of California, San Diego, CA. Chemistry. Ph.D. 2007
\item National Institutes of Health, Bethesda MD. Statistical Mechanics. Postdoc. 2007-2009
\item Argonne National Laboratory, Argonne IL. Biophysics. Postdoc. 2009-2011
\item Duke University. Computational Chemistry. Postdoc. 2011-2013
\end{itemize}

\item Appointments
\begin{itemize}
\item Illinois Institute of Technology. Assistant Professor of Chemistry. 2013-present
\end{itemize}

\item Products
\begin{enumerate}[(i)]
\item Five Products Most Closely Related:
\begin{itemize}
\item Nguyen TH, Zhou HX, Minh DDL. Using the Fast Fourier Transform in Binding Free Energy Calculations. Journal of Computational Chemistry. in press. \url{http://dx.doi.org/10.1002/jcc.25139}
\item Spiridon, L, Minh DDL. Hamiltonian Monte Carlo with Constrained Molecular Dynamics as Gibbs Sampling. J Chem Theory Comput, 13(10), 4649?4659 (2017). % \url{http://dx.doi.org/10.1021/acs.jctc.7b00570}
\item Xie B, Nguyen TH, Minh DDL. Absolute Binding Free Energies between T4 Lysozyme and 141 Small Molecules: Calculations Based on Multiple Rigid Receptor Configurations. J Chem Theory Comput, 13(6), 2930-2944 (2017). %\url{http://dx.doi.org/10.1021/acs.jctc.6b01183}
\item Nguyen TH, Minh DDL. Intermediate Thermodynamic States Contribute Equally to Free Energy Convergence: A Demonstration with Replica Exchange. J Chem Theory Comput, 12(5): 2154-2161 (2016). %\url{http://dx.doi.org/10.1021/acs.jctc.6b00060}
\item Minh DDL, Minh DL, Nguyen L. Layer Sampling. Commun Stat Simulat Comput, 45:1, 73-100 (2016). %\url{http://dx.doi.org/10.1080/03610918.2013.854907}
% \item Minh DDL, Minh DL. Understanding the Hastings Algorithm, Commun Stat Simulat Comput, 44:2, 332-349 (2015). %\url{http://dx.doi.org/10.1080/03610918.2013.777455}
\end{itemize}

\item Five Further Products:
\begin{itemize}
\item Onuk E, Badger J, Wang YJ, Bardhan J, Chishti Y, Akcakaya M, Brooks DH, Erdogmus D, Minh DDL, and Makowski L. Effects of Catalytic Action and Ligand Binding on Conformational Ensembles of Adenylate Kinase. Biochemistry, 56(34), 4559-4567 (2017).
\item Minh DDL. Implicit Ligand Theory: Rigorous binding free energies and thermodynamic expectations from molecular docking.  J Chem Phys, 137: 104106 (2012). %\url{http://dx.doi.org/10.1063/1.4751284}
\item Nilmeier JP, Crooks GE, Minh DDL, Chodera JD. Nonequilibrium candidate Monte Carlo is an efficient tool for equilibrium simulation, Proc Natl Acad Sci USA, 108(45): E1009-1018 (2011).% \url{http://dx.doi.org/10.1073/pnas.1106094108}
\item Minh DDL.  Optimized replica gas estimation of absolute integrals and partition functions.  Phys Rev E, 82(3): 031132 (2010). %\url{http://dx.doi.org/10.1103/PhysRevE.82.031132}
% \item Minh DDL.  Density-Dependent Analysis of Nonequilibrium Paths Improves Free Energy Estimates.  J Chem Phys, 130(20): 204102 (2009). %\url{http://dx.doi.org/10.1063/1.3139189}
\item Minh DDL, Adib AB.  Optimized free energies from bidirectional single-molecule force spectroscopy.  Phys Rev Lett, 100(18): 180602 (2008). %\url{http://dx.doi.org/10.1103/PhysRevLett.100.180602}
\end{itemize}
\end{enumerate}

\clearpage

\item Synergistic Activities
\begin{itemize}
\item I have been extensively involved in mentoring undergraduates and high school students.

IIT undergraduates have been involved in my research group since I started my faculty position. Two students, John Clark and Rachael Youngworth, have completed honors theses with my group. Clark worked on algorithms for clustering receptor snapshots in binding free energy calculations based on implicit ligand theory. Youngworth worked on molecular dynamics simulations of succinate dehydrogenase. Clark now works at Proctor and Gamble and Youngworth is pursing a Ph.D. at the University of Chicago. I currently have two undergraduate students in my group, William Menzer and Natalie Jumonville, who both plan to pursue Ph.D. degrees.

My group has hosted summer internships for 23 undergraduates, 
mostly from the Brazil Scientific Mobility Program.
The students have worked on a variety of projects in small teams mentored by a long-term group member.
One project was a new estimator for thermodynamic quantities based on nonequilibrium fluctuation theorems.
Some projects have involved implementing new features into a software package developed by my group, AlGDock.
These include a Markov chain Monte Carlo method, 
a grid interpolation algorithm, 
and restraints to confine a ligand to a particular pose.
Other projects have been more applied, such as 
constructing homology models, 
running free energy calculations to explore structural ramifications of binding different ligands,
and trying to use BPMFs to predict allostery and functional selectivity.
When I publish the completed projects in the next few years, I plan to include several (but not all) visitors as coauthors.

My group has also hosted 6 high school students.
Like the undergraduates, they have worked in teams mentored by a long-term group member.
They have worked on projects related to molecular docking, 
translating computer code from MATLAB to python,
and predicting residence time based on BPMFs.

While I was a postdoc, I helped mentor a high school and REU student.
The REU student, Clayton Jarratt, actually came to IIT for graduate school and worked with my group for some time.

\item In October 2015, I was the local arrangements chair for the Midwest Enzyme chemistry conference, which was held at IIT. I worked with the program chair to help ensure smooth progress for the free one-day conference for approximately 200 individuals.

\item In the summer of 2015, I helped develop and organize a workshop for high school chemistry teachers entitled 
``Choose Your Own Adventure: Solving Real-World Problems with Spectroscopy.''
After a lecture about spectroscopic techniques and a tour of instruments, 
the teachers devised and implemented laboratory-based solutions to a set of realistic scenarios.
The teachers responded enthusiastically, asking many questions 
and performing careful sample preparation and measurement.
One teacher effused that if we held the workshop again, 
he would return to work on scenarios that he missed the first time!

\item Much of my research can be described as development and refinement of research tools, computation methodologies, and algorithms for problem-solving. Throughout my scientific career, I have made source code publicly available on the internet.

\item I have reviewed 33 articles for 15 different journals: ACS Omega (1), Biophysical Journal (2), Biopolymers (2), Journal of Chemical Physics (10), Journal of Chemical Information and Modeling (2), Journal of Chemical Theory and Computation (5), Journal of Computer-Aided Molecular Design (1), Journal of Molecular Biology (1), Journal of Physical Chemistry (3), Journal of Statistical Physics (1), Journal of Structural Biology (1), Langmuir (1), Physics Letters A (1), Physical Biology (1), and PloS One (1). The number in parentheses is the number of articles reviewed for each journal.
\end{itemize}

\end{enumerate} 