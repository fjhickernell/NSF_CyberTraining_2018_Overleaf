\documentclass[11pt]{NSFamsart}
\usepackage[utf8]{inputenc}
\usepackage{xspace}
\usepackage[dvipsnames]{xcolor}
\usepackage[numbers]{natbib}
\usepackage{hyperref,array,accents,longtable,booktabs}
\usepackage{cleveref}
%\usepackage[notref,notcite]{showkeys} %This package prints the labels in the margin

\setlength{\leftmargini}{2.5ex} %indentation of the left margin of the itemize


\thispagestyle{plain}
\pagestyle{plain}

\headsep-0.6in
\textwidth6.5in
\oddsidemargin0in
\evensidemargin0in
\textheight9in

\newcommand{\myshade}{85}
\colorlet{mylinkcolor}{violet}
\colorlet{mycitecolor}{Aquamarine}
\colorlet{myurlcolor}{YellowOrange}
\hypersetup{ %make the links stand out
	linkcolor  = mylinkcolor!\myshade!black,
	citecolor  = mycitecolor!\myshade!black,
	urlcolor   = myurlcolor!\myshade!black,
	colorlinks = true,
}

\providecommand{\FJHickernell}{Hickernell}

% Everyone feel free to add their own note definitions here
\newcommand{\FJHNote}[1]{{\textcolor{blue}{FJH: #1}}}
\newcommand{\DMNote}[1]{{\textcolor{green}{DM: #1}}}
\newcommand{\JWNote}[1]{{\textcolor{orange}{JW: #1}}}
\newcommand{\SCCNote}[1]{{\textcolor{magenta}{SCC: #1}}}
\newcommand{\XHSNote}[1]{{\textcolor{red}{XHS: #1}}}
\newcommand{\NLNote}[1]{{\textcolor{yellow}{NL: #1}}}
\newcommand{\KWONote}[1]{{\textcolor{cyan}{KWO: #1}}}



\newcommand{\JW}{Wereszczynski\xspace} %help me spell his name correctly
\newcommand{\Order}{\mathcal{O}}









\begin{document}
\leftmargini2.5ex %indentation of the left margin of the itemize

[Draft Title:]  	CyberTraining: CIC:  Educating the Next Computational Scientists

\noindent
\XHSNote{A note from Xian-He}\\
\DMNote{A note from David} \\
\JWNote{A note from Jeff} \\
\SCCNote{A note from Sou-Cheng} \\

\bigskip


\bigskip

\centerline{\Large \textbf{Project Description}}
\vspace{-2ex}

\setcounter{tocdepth}{1}
\tableofcontents %Help the readers navigate the proposal

\vspace{-6ex}

\section{Introduction}
Computational science draws upon the strengths of several disciplines, including computer science, mathematical science, and natural science.  Realizing the full potential of computational science for discovery requires multidisciplinary teams whose members can take full advantage of advanced hardware architectures and software environments.  Computational scientists require

\begin{itemize}
\item Depth within their chosen disciplines plus breadth across relevant disciplines, overcoming a silo mentality, and
\item Experience with advanced cyberinfrastructure (CI), including the practices that extend beyond what is required for computation on a single CPU.
\end{itemize}

The newly established Center for Interdisciplinary Scientific Computation (CISC) \url{http://cos.iit.edu/cisc} at Illinois Institute of Technology will lead the development of a new program to educate computational scientists that have the above attributes.  We will educate high school through doctoral students.  Learning will be curricular, co-curricular, and extra-curricular.  We will partner with nearby national laboratories, companies engaged in advanced computing, and schools whose students have less access to research experience and high-end computing facilities.

\subsection*{The Challenges}
Computing solutions to complex scientific problems requires well-educated computational scientists.  Successful attempts to train computational scientists must overcome two kinds of challenges.

\subsubsection*{Academic Silos.} Computational science draws on multiple disciplines, including computer science, mathematical science, and natural sciences.  Students must gain a a deep understanding of their chosen discipline:

{\color{green}[DM: Should this sentence anticipate going beyond the single discipline?]}

\begin{itemize}
\item Computer scientists must understand how emerging languages and architectures enable faster and more scaleable computation.  

\item Computational mathematicians must understand how to describe and analyze the errors in numerical algorithms and statistical models.

{\color{green}[DM: Is there anything beyond error? Perhaps develop more efficient algorithms for different classes of applications?]}

\item Computational biologists, chemists, and physicists must grasp the important scientific ideas that need to be captured by computations.  

{\color{green}[DM: Perhaps ``Must understand the benefits and limitations of using computer modeling in their respective disciplines''. Also, do we want to mention engineers?]}

\end{itemize}
Good computational scientists must help grow their own disciplines.  However, at the same time, they must be conversant in these other disciplines so that they can be effective members of  interdisciplinary teams tackling large, complex computational science problems.

{\color{green}[DM: Is this an imperative? As a counterexample, a computational chemist can use existing software tools to make testable hypotheses related to chemistry without interacting with mathematicians or computer scientists. Perhaps it would be more accurate to reverse the logic of the sentence to say ``To be effective members of interdisciplinary teams tackling large, complex computational science problems, computational scientists must be conversant in other disciplines.'']}

\subsubsection*{Ignorance of Good Practice for Large Scale Computation.}  Solving large, complex computational problems using advanced CI deviates in important ways from what students learn in their coursework. 

\begin{itemize}

\item Effective algorithms that take advantage of muti-core, distributed memory architectures may be substantially different than algorithms designed for single CPUs.

\item The software required to solve large problems is often drawn from multiple sources, written in several languages, and developed over the course of years.

\item Software users must ensure that their results can be reproduced by others who come after them \cite{Pen11}.  

\item Software contributors must ensure that their software is robust, can interact well with other software, and can be extended by those who come afterwards.  See \cite{BSS18} for a recent effort to promote these principles broadly.

\end{itemize}

This proposal describes how we will overcome these challenges in educating the next generation of computational scientists.

\subsection{Why CISC Is Particularly Poised to Lead This}
CISC was created in May 2017 to leverage Illinois Tech’s existing strengths in computational science for greater impact.  CISC has been given office space, a modest budget, a 256-core cluster, and the good will of supporters among faculty and administrators.  Since its inception, CISC has initiated a series of lunchtime matchmaking seminars, brought its cluster online, hosted special lectures, and sponsored a seed grant competition. Fred J. Hickernell (FJH), PI and CISC’s director, is one of several computational mathematicians at Illinois Tech and previously served for twelve years as chair of the applied (and only) mathematics department.  Our computer science department has a historically strong group in high performance computing led by Xian-He Sun (XHS), a co-PI.  In recent years the biology, chemistry, and physics departments have hired computational scientists, including co-PIs David Minh (DM) (chemistry, CISC's assodiate director) and Jeff Wereszczynski (JW) (physics).  Co-PI Sou-Cheng Choi is a lead researcher at Allstate and a research associate professor in applied mathematics.

The (co-)PIs have experience developing curricula, mentoring high school students through PhD students, and partnering with College of DuPage, Argonne National Laboratory, and Fermilab.  Our proposed innovations will build upon our track record to build a strong multi-faceted training program for computational scientists.

\section{Results from Prior NSF Support} \FJHNote{Everyone needs to fill out their previous experience, need subsections on intellectual merit and broader impacts}
\subsection{Experience of FJH and SCC}
NSF-DMS-1522687\except{toc}{, \emph{Stable, Efficient, Adaptive Algorithms for Approximation and 
Integration}, \$270,000, August 2015 -- July 2018.} \label{SectFJHPrevious}  FJH is PI and SCC is senior personnel.  

\subsubsection{Intellectual Merit}
One of the primary outcomes of this project related to the present proposal is the development of the Guaranteed Automatic Integration Library (GAIL) \cite{ChoEtal17b}.  This library comprises univariate and multivariate integration, univariate function approximation, and univariate optimization algorithms that automatically determine the sample size required to meet user-defined error tolerances.  The library does not rely on interval arithmetic, as is done in INTLAB \cite{MoKeCl09, Rum99a, Rum10a}, but like INTLAB, GAIL comes with theoretical guarantees that common adaptive algorithms lack. The recent GAIL developments include locally adaptive function approximation and optimization \cite{ChoEtal17a, Din15a}, adaptive quasi-Monte Carlo cubature \cite{HicJim16a, JimHic16a}, and the ability to set a hybrid error tolerance involving both absolute and relative error criteria \cite{HicEtal17a}.  Other Articles, theses,  software, and preprints supported in part by this grant include 
\cite{ala_augmented_2017, 
	GilEtal16a,
	GilJim16b,
	HicEtal18a,	
	Hic17a,
	JohFasHic18a,
	Li16a,
	Liu17a,
	mccourt_stable_2017,
	mishra_hybrid_nodate,
	mishra_stable_nodate, 
	rashidinia_stable_nodate,
	vu_rbf-fd_nodate,
	Zha17a,
	Zho15a,
	ZhoHic15a}.
    
\subsubsection{Broader Impacts}  Three students (two female) have completed their PhD degrees, and three PhD students are in the midst of their PhD degrees.  One PhD graduate spent time working at Fermilab as a student helping them implement modern Monte Carlo methods.  Another PhD student is picking up where the first one left off.  Two students (one female) completed their MS theses, and one female student has nearly completed her MS thesis. More than a dozen undergraduate students have been mentored (primarily during the summer),  over the course of three summers.  FJH has embedded the new adaptive (quasi-)Monte Carlo research in his yearly graduate Monte Carlo class, and some students in that class have contributed to GAIL.  FJH and SCC and their students have written an encyclopedia article, given numerous conference and colloquium talks, and organized a conference and special conference sessions at multiple conferences.  FJH has given an invited conference tutorial and is one of the program leaders for this year's SAMSI program on quasi-Monte Carlo sampling.  FJH received the 2016 Joseph F.\ Traub Prize for Achievement in Information-Based Complexity.






\subsection{Experience of XHS}

\subsection{Experience of DM}

DM is a new investigator and has not previously received NSF funding.

\DMNote{Fred, should I still say something about my research (unfunded and NIH funded) and broader impacts?}

\subsection{Experience of JW}



\section{Intellectual Merit}

\subsection{Skills to Be Learned}
Our primary emphasis is educating CI contributors (CIC), with a secondary emphasis on educating CI users (CIU).  Our goal is to train students from high school through PhD level in skills of increasing difficulty.  These include the following

\begin{description}
\item[CIU] Write and run numerical programs on a single CPU \DMNote{Isn't this CIC?}
\item[CIU] Write and run numerical programs that take advantage of multiple cores and/or a GPU on a single machine \DMNote{Isn't this CIC?}
\item[CIU/CIC] Write and run multiple jobs simultaneously to a cluster
\item[CIU/CIC] Write and run jobs on a cluster that each require multiple cores with tight connectivity
\item[CIU/CIC] Write and run jobs on an advanced machine 
\item[CIC] Use and write tools for analysis and visualization of large scale simulations
\item[CIC] Develop a well-documented numerical software library consisting of tested routines
\item[CIC] Develop or contribute to a numerical software library takes advantage of high performance computing architectures
\item[CIC] Analyze the computational efficiency of algorithms and identify performance bottlenecks
\item[CIU/CIC] Translate scientific models into mathematical formulations and then into computer code
\item[CIC] Evaluate whether simulation output accurately reflects the natural phenomenon it is designed to emulate.  


\end{description}






\subsection{Innovations}
Our proposed innovations are targeted 


\subsubsection*{Stronger high performance(?) science coursework across the science curriculum} At present Illinois Tech has  \FJHNote{list what we have already} 
\begin{itemize} 

\item \FJHNote{computer science offerings}

\item A computational mathematics specialization within the BS Applied Mathematics, four undergraduate computational mathematics courses (one required), and five(?) graduate computational mathematical courses

\end{itemize}

However, 

We wa
\subsubsection*{New degrees or certificates???}

\subsubsection*{Summer camps for high school students} For several years Illinois Tech's College of Science has run a three-week computational science summer camp for Chicago area high school students.  The goal has been to introduce them to solving mathematical and scientific problems using Mathematica.  The Admissions Office publicizes and recruits students for the camp.  Starting in the summer of 2018 this summer camp has an academic home in CISC, while continuing to partner with the Admissions Office.

We propose to broaden this camp to include computing on the CISC cluster to introduce these high school students to multi-core computing.  This will encourage them to pursue degrees involving high performance computation and better prepare them for careers as computational sciences.  We also hope to attract them to Illinois Tech's computational science offerings


\subsubsection*{Post-doctoral scholars???} 
\subsubsection*{Residencies for graduate students in other research groups}
\subsubsection*{Sharing of coursework across majors}
\subsubsection*{Summer internships} 
\subsubsection*{Research experiences for under-served undergraduate students} \FJHNote{Jeff please write this}
\subsubsection*{Summer camps for high school students}


Software is important \cite{RudEtal18a}

\section{Broader Impacts}
\section{Evaluation}


\newpage
\clearpage
%\pagenumbering{arabic}
\setcounter{page}{1}

\bibliographystyle{spbasic}

{\renewcommand\addcontentsline[3]{} 
\renewcommand{\refname}{{\Large\textbf{References Cited}}}                   %%
\renewcommand{\bibliofont}{\normalsize}

\bibliography{FJH23,FJHown23,GregPapers}}

\end{document}

Review-specific criteria
1. Challenges addressed in training, education, and workforce development;
2. New modes of discovery and use of advanced CI resources, tools, and services in fundamental research enabled;
3. Advances in integrating skills in advanced CI as well as computational and data science and engineering into institutional and disciplinary curriculum/instructional material;
4. Steps to broaden access and community adoption with respect to the Nation’s scientific and engineering research workforce and advanced CI;
5. Stakeholders engaged and partnerships forged for collective impact;
6. Scalability to a large number of people directly and indirectly, and sustainability of key aspects beyond NSF funding; and
7. Plans for recruitment and assessment.

Proposals must clearly address the following solicitation-specific review criteria through well-identified proposal elements:
1. Are the training, education, and research workforce challenges identified sound?
2. What is the potential of the project to enable new modes of discovery and use of advanced CI resources, tools, and services in fundamental research?
3. How well would the project advance the goal of integrating skills in advanced CI as well as computational and data science and engineering into institutional and disciplinary curriculum/instructional material?
4. To what extent can the project meet its broadening access and community adoption challenges with respect to the Nation’s scientific and engineering research workforce and advanced CI?
5. How well would the project engage key stakeholders and forge partnerships for collective impact?
6. What is the potential for the project to scale and for its key aspects to be sustained beyond NSF funding?
7. Are the plans for recruitment and evaluation sound?
8. Are the plans for management and collaboration effective?

