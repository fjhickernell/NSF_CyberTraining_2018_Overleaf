\documentclass[11pt]{NSFamsart}
\usepackage[utf8]{inputenc}
\usepackage{xspace}
\usepackage[dvipsnames]{xcolor}
\usepackage[numbers]{natbib}
\usepackage{hyperref,array,accents,longtable,booktabs}
\usepackage{cleveref}
%\usepackage[notref,notcite]{showkeys} %This package prints the labels in the margin

\setlength{\leftmargini}{2.5ex} %indentation of the left margin of the itemize


\thispagestyle{plain}
\pagestyle{plain}

\headsep-0.6in
\textwidth6.5in
\oddsidemargin0in
\evensidemargin0in
\textheight9in

\newcommand{\myshade}{85}
\colorlet{mylinkcolor}{violet}
\colorlet{mycitecolor}{Aquamarine}
\colorlet{myurlcolor}{YellowOrange}
\hypersetup{ %make the links stand out
	linkcolor  = mylinkcolor!\myshade!black,
	citecolor  = mycitecolor!\myshade!black,
	urlcolor   = myurlcolor!\myshade!black,
	colorlinks = true,
}

\providecommand{\FJHickernell}{Hickernell}

% Everyone feel free to add their own note definitions here
\newcommand{\FJHNote}[1]{{\textcolor{blue}{FJH: #1}}}
\newcommand{\DMNote}[1]{{\textcolor{green}{DM: #1}}}
\newcommand{\JWNote}[1]{{\textcolor{orange}{JW: #1}}}
\newcommand{\SCCNote}[1]{{\textcolor{magenta}{SCC: #1}}}
\newcommand{\XHSNote}[1]{{\textcolor{red}{XHS: #1}}}
\newcommand{\NLNote}[1]{{\textcolor{yellow}{NL: #1}}}
\newcommand{\KWONote}[1]{{\textcolor{cyan}{KWO: #1}}}



\newcommand{\JW}{Wereszczynski\xspace} %help me spell his name correctly
\newcommand{\Order}{\mathcal{O}}







\newcounter{skillct}
\newcommand{\skillnum}[1]{\refstepcounter{skillct}\label{#1}\arabic{skillct}.}
\newcommand{\notyet}{\textbf{?}}
\newcommand{\done}{\checkmark}




\begin{document}
\leftmargini2.5ex %indentation of the left margin of the itemize
\centerline{\Large \textbf{Contents of This Document}}
\bigskip
\begin{itemize}
\item To Do List
\item Project Description
\item References
\item Management and Coordination Plan
\item Data Management
\item Budget Justification
\end{itemize}

All documents are in one so that the references to different sections can be supported.

\newpage \setcounter{page}{1} %%%%%%%%%%%%%%%%%%%%%%%%%%%%%%%%%%%%%%%%%%%%%%%%%%

\centerline{\Large \textbf{To Do List}}
\bigskip

\begin{itemize}
\item[\done] Confirm Title: CyberTraining: CIC:  Cross-Disciplinary Education for Next-Generation Computational Scientists
\item Project Description
\begin{itemize}
\item Previous funding
\begin{itemize}
\item[\done] Fred
\item[\notyet] Xian-He
\item[\done] David
\item[\done] Jeff
\item[\done] Sou-Cheng
\item[\notyet] Norm
%\item[\notyet] Kiah Wah
\end{itemize}
\item[\done] References (Jeff)
%\item[\notyet] Graduate course in computational natural science (David, Jeff)
\item[\notyet] Broader impacts
\item[\notyet] Evaluation (Norm)
\end{itemize}

\item[\notyet] Budget

\item NSF Biosketches
\begin{itemize}
\item[\notyet] Fred
\item[\done] Sou-Cheng
\item[\done] Xian-He
\item[\done] David
\item[\notyet] Jeff
\item[\notyet] Norm
\item[\notyet] Kiah Wah
\end{itemize}


\item Collaborators and other affiliations
\begin{itemize}
\item[\notyet] Fred
\item[\done] Sou-Cheng %\SCCNote{Sent to Brian Davis  at \texttt{jdavis10@iit.edu} on Feb 9, 2018. A copy, \texttt{20180209\_Sou\_Cheng\_Choi\_COA.xlsx}, is uploaded to Overleaf's project folder \texttt{Biosketch}.}
\item[\notyet] Xian-He
\item[\done] David
\item[\notyet] Norm
\item[\notyet] Kiah Wah
\end{itemize}

\item Facilities
\begin{itemize}
\item[\notyet] Xian-He's lab
\end{itemize}

\item Letters of Collaboration
\begin{itemize}
\item[\notyet] Lois
\item[\done] Burt
\item[\done] COD
\item[\notyet] April Welch
\end{itemize}

\item[\notyet] Data Management

\item[\notyet] Management
\end{itemize}

\bigskip

\noindent
\XHSNote{A note from Xian-He}\\
\DMNote{A note from David} \\
\JWNote{A note from Jeff} \\
\SCCNote{A note from Sou-Cheng} \\
\NLNote{A note from Norm Lederman}\\
\KWONote{A note from KiahWahOng}\\

\newpage \setcounter{page}{1} %%%%%%%%%%%%%%%%%%%%%%%%%%%%%%%%%%%%%%%%%%%%%%%%%%


% Review-specific criteria
% 1. Challenges addressed in training, education, and workforce development;
% 2. New modes of discovery and use of advanced CI resources, tools, and services in fundamental research enabled;
% 3. Advances in integrating skills in advanced CI as well as computational and data science and engineering into institutional and disciplinary curriculum/instructional material;
% 4. Steps to broaden access and community adoption with respect to the Nation’s scientific and engineering research workforce and advanced CI;
% 5. Stakeholders engaged and partnerships forged for collective impact;
% 6. Scalability to a large number of people directly and indirectly, and sustainability of key aspects beyond NSF funding; and
% 7. Plans for recruitment and assessment.

% Proposals must clearly address the following solicitation-specific review criteria through well-identified proposal elements:
% 1. Are the training, education, and research workforce challenges identified sound?
% 2. What is the potential of the project to enable new modes of discovery and use of advanced CI resources, tools, and services in fundamental research?
% 3. How well would the project advance the goal of integrating skills in advanced CI as well as computational and data science and engineering into institutional and disciplinary curriculum/instructional material?
% 4. To what extent can the project meet its broadening access and community adoption challenges with respect to the Nation’s scientific and engineering research workforce and advanced CI?
% 5. How well would the project engage key stakeholders and forge partnerships for collective impact?
% 6. What is the potential for the project to scale and for its key aspects to be sustained beyond NSF funding?
% 7. Are the plans for recruitment and evaluation sound?
% 8. Are the plans for management and collaboration effective?


\centerline{\Large \textbf{Project Description}}
\vspace{-2ex}

\setcounter{tocdepth}{2}
\tableofcontents %Help the readers navigate the proposal

\vspace{-6ex}

\section{Introduction}
Computational sciences typically draw upon the knowledge and strengths of several disciplines, including computer science, mathematical science, and natural sciences.  Realizing the full potential of computational science for discovery requires multidisciplinary teams whose members can take full advantage of advanced hardware architectures and software environments.  Computational scientists often require the following qualities to conduct effective research:

\begin{itemize}
\item depth within their chosen disciplines plus breadth across relevant disciplines, overcoming a silo mentality, and
\item experience with advanced cyberinfrastructure (CI), including the practices that extend beyond what is required for computations on a single processor or a core  in a multi-core CPU (central processing unit).
\end{itemize}

The newly established Center for Interdisciplinary Scientific Computation (CISC) \url{http://cos.iit.edu/cisc} at Illinois Institute of Technology (IIT, or Illinois Tech) will lead the development of a new program \DMNote{a program or a few programs? ``development of new programs''?} to educate computational scientists that have the two attributes above.  We will educate high school through doctoral students.  Learning will be curricular and extra-curricular.  We will partner with nearby national laboratories, companies engaged in advanced computing, and schools whose students have less access to research experience and high-end computing facilities.

\subsection*{The Challenges}
Computing solutions to complex scientific problems requires properly educated computational scientists.  This proper education must overcome two kinds of challenges.

\subsubsection*{Academic Silos.} Computational science draws on multiple disciplines, including computer science, mathematical science, and natural sciences.  Students must gain a deep understanding of their chosen discipline. Computational scientists should not be a jack of all trades and a master of none.

\begin{itemize}
\item Computer scientists must understand how emerging languages and architectures enable faster and more scalable computations.  

\item Computational mathematicians must be able to describe and analyze truncation and round-off errors in numerical algorithms, as well as measurement and misspecification errors of statistical models.  They must know how to choose the most efficient algorithms for different kinds of problems.

\item Computational biologists, chemists, and physicists must grasp the important scientific ideas that need to be captured by computations.  They must understand the benefits and limitations of using computer modeling in their respective disciplines.

\end{itemize}
But this is not enough!  Computational scientists should be a master of one trade and conversant with others.  To be effective members of interdisciplinary teams tackling large, complex computational science problems, computational scientists must be ready to exit their silos and become familiar with other disciplines.

\subsubsection*{Ignorance of Good Practice for Large Scale Computation.}  Solving large, complex computational problems using advanced CI deviates in important ways from what students typically learn in their coursework. See \cite[Sect.\ 2.4]{RudEtal18a} for a more detailed discussion.  Here are some highlights:

\begin{itemize}

\item Effective algorithms that take advantage of muti-core, distributed memory architectures may be substantially different than algorithms designed for a single CPU with only one core.

\item The software required to solve large problems is often drawn from multiple sources, written in several languages, and developed over the course of years, by different experts.

\item Those who perform computational experiments and those who demonstrate the performance of their new algorithms must ensure that their results can be \emph{reproduced} by others \cite{Pen11}.  

\item Software contributors must ensure that their software is robust, interacts well with other software, and can be extended by those who come afterwards.  See \cite{BSS18} for a recent effort to promote these principles broadly.

\end{itemize}

We propose to overcome these challenges in educating the next generation of computational scientists.

\subsection*{Why CISC is Poised to Lead This}
CISC was created in May 2017 to leverage Illinois Tech’s existing strengths in computational science for greater impact.  CISC has been given office space, a modest budget, a 256-core cluster, and the good will of supporters among faculty and administrators.  Since its inception, CISC has initiated a series of lunchtime matchmaking seminars \SCCNote{matchmaking means?}, brought its cluster online, hosted special lectures, and sponsored a seed grant competition. Fred J. Hickernell, PI and CISC’s director, is one of several computational mathematicians at Illinois Tech and previously served for twelve years as chair of the applied (and only) mathematics department.  Our computer science department has a historically strong group in high performance computing led by Xian-He Sun, a co-PI.  In recent years, the biology, chemistry, and physics departments have hired computational scientists, including co-PIs David Minh (chemistry, CISC's associate director) and Jeff Wereszczynski (physics).  Co-PI Sou-Cheng Choi is a lead researcher at Allstate Insurance Company's Automotive and Life/Retirement innovation teams, and also a research associate professor in applied mathematics at Illinois Tech.

The (co-)PIs have experience developing curricula, mentoring high school students through PhD students, and partnering with College of DuPage, Argonne National Laboratory, and Fermilab.  Our proposed innovations will build upon our track record to build a strong multi-faceted training program for computational scientists.

\section{Results from Prior NSF Support} \FJHNote{Everyone needs to fill out their previous experience, need subsections on intellectual merit and broader impacts}
\subsection{Experience of Hickernell and Choi}
NSF-DMS-1522687, \emph{Stable, Efficient, Adaptive Algorithms for Approximation and 
Integration}, \$270,000, August 2015 -- July 2018. \label{SectHickernellPrevious}  Hickernell is PI and Choi is senior personnel.  

\subsubsection{Intellectual Merit}
One of the primary outcomes of this project in relation to the present proposal is the development of the Guaranteed Automatic Integration Library (GAIL) \cite{ChoEtal17b}.  This library comprises univariate and multivariate integration, univariate function approximation, and univariate optimization algorithms that automatically determine the sample size required to meet user-defined error tolerances.  The library does not rely on interval arithmetic, as is done in INTLAB \cite{MoKeCl09, Rum99a, Rum10a}, but like INTLAB, GAIL comes with theoretical guarantees that common adaptive algorithms lack. The recent GAIL developments include locally adaptive function approximation and optimization \cite{ChoEtal17a, Din15a}, adaptive quasi-Monte Carlo cubature \cite{HicJim16a, JimHic16a}, and the ability to set a hybrid error tolerance involving both absolute and relative error criteria~\cite{HicEtal17a}.  Other Articles, theses,  software, and preprints supported in part by this grant include 
\cite{ala_augmented_2017, 
	GilEtal16a,
	GilJim16b,
	HicEtal18a,	
	Hic17a,
	JohFasHic18a,
	Li16a,
	Liu17a,
	mccourt_stable_2017,
	mishra_hybrid_nodate,
	mishra_stable_nodate, 
	rashidinia_stable_nodate,
	vu_rbf-fd_nodate,
	Zha17a,
	Zho15a,
	ZhoHic15a}.
    
\subsubsection{Broader Impacts}  Three students (two female) have completed their PhD degrees, and three PhD students are in the midst of their PhD degrees.  One PhD graduate spent time working at Fermilab as a student helping them implement modern Monte Carlo methods.  Another PhD student is picking up where the first one left off.  Two students (one female) completed their MS theses, and one female student has nearly completed her MS thesis. More than a dozen undergraduate students have been mentored (primarily during the summer),  over the course of three summers.  Hickernell has embedded the new adaptive (quasi-)Monte Carlo research in his yearly graduate Monte Carlo class, and some students in that class have contributed to GAIL.  Hickernell, Choi, and their students have written an encyclopedia article, given numerous conference and colloquium talks, and organized a conference and special conference sessions at multiple conferences.  Hickernell has given an invited conference tutorial and is one of the program leaders for this year's SAMSI program on quasi-Monte Carlo sampling. Hickernell received the 2016 Joseph F.\ Traub Prize for Achievement in Information-Based Complexity.

\subsection{Experience of Minh}

DM is a new investigator and has not previously received NSF funding.  

% \DMNote{Fred, should I still say something about my research (unfunded and NIH funded) and broader impacts?}

% \FJHNote{David, yes, please say something.}


\subsection{Experience of Sun}  NSF-CNS-1162540, \emph{CSR: Medium: Collaborative Research: Decoupled Execution Paradigm for Data-Intensive High-End Computing,} \$280,766, June 2012 -- August 2016. Sun is the PI.  
\subsubsection{Intellectual Merit} This collaborative project (total budget \$842,298.00, shared with Prof.~W.\ Gropp of U Illinois and Prof.~Y.~Chen of Texas Tech U developed a decoupled execution paradigm (DEP) to address I/O bottleneck issues. This paradigm enables users to identify and handle data-intensive operations separately. Broader Impacts: Technical hurdles have been identified for decoupled execution, spanning system architecture, programming model, and runtime system. Results are available as recent publications \cite{HeWS16, HeWS16a, HWSX17, HWLS17, HWSH17, KYES16, SuWa14, WaSu14, ZLKR16} and other I/O related publications. In addition, in the past five years, Dr. Sun has graduated 3 PhD and 11 masters students and supervised 5 postdoctoral and 4 graduate researchers.



\subsection{Experience of Wereszczynski}

%\JWNote{I have two NSF grants to report on.  Note sure how you want to arrange them}

PI of NSF-MCB-1552743   ``CAREER: The Effects of Post-translational Modifications and Histone Variants on Chromatin Fiber Dynamics'' June 2016 -- May 2021, \$790,129.

\subsubsection{Intellectual Merit} The aim of this CAREER award is to use and develop multiscale biophysical simulation techniques to study how DNA is compacted in the cell, and how these mechanisms are regulated to affect gene expression.  The PI's group addresses these issues through the use of molecular dynamics (MD) simulations, along with collaborations at Argonne National Laboratory and the University of Iowa.  Early results have been presented at the annual Biophysical Society Meeting in 2017 and 2018, and a manuscript detailing the mechanisms by which post-translational modifications to the ``H3 tails'' affect the structures and energetics of nucleosomes is under minor review at \textit{eLife.} \SCCNote{Add a citation?}

\subsubsection{Broader Impacts} To date, this proposal has funded four trainees: one postdoctoral scholar, one PhD student, and two undergraduates from the College of DuPage (COD).  Both undergraduates performed 10-week summer research projects, during which time they learned how to set up, perform, and analyze MD simulations of various biomolecular systems. Mr.~Robert Hickok, who was in our lab in the summer of 2016, is currently enrolled at the University of Illinois at Chicago, whereas Mr.~Meet Patel, who worked with us in 2017, is enrolled at Georgia Institute of Technology.  


%\subsubsection*{NSF-MCB-1716099
%Collaborative Research: Molecular Mechanism of Heme Extraction by IsdH''   August 2017-July 2020 \$250,000 (collaborative with Robert T. Clubb, UCLA, \$700,000 total budget)}
%\textbf{Intellectual Merit}  This is a newly awarded proposal that supports an ongoing collaboration between our group and the Clubb group at UCLA.  The aim of this proposal is to use experimental and computational techniques to understand the mechanisms by which the bacterial protein IsdH scavenges iron from host hemoglobin.  Initial results of our modeling results will be presented at the 2018 Biophysical Society Meeting and a join experimental and computational manuscript is currently under minor revision publication in the Journal of Biological Chemistry.

%\textbf{Broader Impacts} To date, only one trainee has been supported by this proposal: Joseph Clayton, a PhD student in our group.  It is expected that in the summer of 2018 another undergraduate student from CoD will be supported on this award.

\section{Intellectual Merit}

\subsection{Skills to Be Learned}
Our primary emphasis is educating CI contributors (CICs), and our secondary emphasis is educating CI users (CIUs).  Our goal is to train students from high school through PhD level in skills of various difficulty, some building upon others.  These include the following.

\begin{longtable}
{r>{\raggedright}p{0.62\textwidth}ccc}
\multicolumn{2}{l}{\textbf{Learning Outcomes}} & \textbf{CIU} & \textbf{CIC} & \textbf{Learned}\\
\toprule
\multicolumn{2}{l}{\emph{Use}} \tabularnewline
\skillnum{OneCPU} & Write and run numerical programs on a single CPU &
HS & HS & \ref{Camp}, \ref{CurrExist}, \ref{CODSummer} \tabularnewline
\skillnum{TwoCPU} & Run numerical programs that utilize multiple cores and/or a GPU on a single machine &
UG & HS & \ref{Camp}\tabularnewline
%\skillnum{MultiJobs} & Run multiple jobs simultaneously on a cluster &
% UG & HS & \ref{Camp} \tabularnewline
\skillnum{Tight} & Run jobs that require more than one node with tight connectivity &
G & UG & \ref{Fellow} \tabularnewline
\skillnum{TopMach} & Run jobs on a top-500 machine &
G & UG & \ref{Fellow} \tabularnewline
\skillnum{AnalVis} & Use tools for analysis and visualization of large scale simulations &
G & G & \ref{LargeSC}, \ref{Fellow} \tabularnewline
\skillnum{MultiLing} & Solve scientific problems that require multiple libraries and languages & G & UG & \ref{Fellow} \tabularnewline
\skillnum{Repro} & Execute reproducible scientific computations & UG & UG & \ref{RelSoft} \tabularnewline
\midrule
\multicolumn{2}{l}{\emph{Develop \& Optimize}} \tabularnewline
\skillnum{TwoCPUWrite} & Write numerical programs that take advantage of multiple cores and/or a GPU on a single machine &
UG & HS & \ref{Camp}\tabularnewline
\skillnum{ContLib} & Make additions to a well-documented numerical software library consisting of documented, tested, robust routines & & G & \ref{RelSoft} \tabularnewline
\skillnum{HighLib} & Contribute to a numerical software library that takes advantage of high performance computing architectures &
 & G & \ref{Fellow} \tabularnewline
%\midrule
%\multicolumn{2}{l}{\textbf{Optimize}} \tabularnewline
\skillnum{EffOne} & Analyze the computational efficiency of individual algorithms and identify performance bottlenecks &
& UG & \ref{CurrExist} \tabularnewline
\skillnum{EffLarge} & Analyze the computational efficiency of large scale simulations and identify performance bottlenecks &
& G & \ref{Fellow}\tabularnewline
\midrule
\multicolumn{2}{l}{\emph{Apply}} \tabularnewline
\skillnum{Natural} & Evaluate whether simulation output accurately reflects the natural phenomenon it is designed to emulate &
UG & UG & \ref{CurrExist}, \ref{Fellow} \tabularnewline
\skillnum{Decision} & Appreciate how a computation informs decision-making in the application domain and its effect on the kind of computation needed & G & G & \ref{Fellow} \tabularnewline
\skillnum{Model} & Know how different modeling assumptions in the application domain are tied to the choice of different computational methods and computing environments & G & G & \ref{Fellow} \tabularnewline
\skillnum{Other} & Collaborate with computational scientists outside their own majors & UG & UG & \ref{Fellow} \tabularnewline
& \FJHNote{What are we missing?  Please add} \tabularnewline
\bottomrule
\end{longtable} 

Many of the learning outcomes above are parallel to those given in \cite[Table 1]{RudEtal18a}.  \DMNote{Upon seeing this citation I originally thought that it was referring to a table within the grant proposal. It might be a little bit confusing.} However, that table focuses on the outcome for a PhD in computational science.  Our table above provides intermediate outcomes for those in high school, bachelor's, and master's study.  A future CIC should ideally develop all of the skills labeled HS in high school, all of the skills labeled UG during undergraduate studies, and all of the skills labeled G during graduate studies.  The analogous interpretation applies to CIUs.  Our proposed curricular and extra-curricular innovations, outlined in the remainder of this section, provide opportunities to future CICs and CIUs to develop these skills.  

The degree of mastery of any particular skill may depend on whether the computational scientist is majoring in computer science, mathematical science, or natural science.  Moreover, our students may only be with us for part of their high school through PhD studies.  We are ready to help those whose preparation is lacking to catch up.  When students leave our program they should be fully prepared for the next steps in their growth as computational scientists.

\subsection{Program Innovations}
\newcommand{\CampName}{Summer computational science course}
\newcommand{\CurrExistName}{Increasing accessibility to existing computational science offerings}
\newcommand{\RelSoftName}{Professional practices for computational science course} 
\newcommand{\LargeSCName}{Large-scale scientific computation course}
\newcommand{\PhyChemBioCompSciName}{New computational natural sciences course}
\newcommand{\CODSummerName}{Research experiences for under-served undergraduates}
\newcommand{\FellowName}{CISC summer fellowships}

Our innovations will better prepare CICs and CIUs via several modes of learning for different kinds of students.  They can be summarized as follows:

\begin{center}
\begin{tabular}
{lcc}
Innovation & \multicolumn{2}{c}{Target} \tabularnewline
\toprule
\CampName & CIU & CIC \tabularnewline
\CurrExistName & CIU & CIC \tabularnewline
\RelSoftName &  & CIC \tabularnewline
\LargeSCName &  & CIC \tabularnewline
%\PhyChemBioCompSciName & CIU & CIC \tabularnewline
\CODSummerName & CIU & CIC \tabularnewline
\FellowName &  & CIC \tabularnewline
\bottomrule
\end{tabular}
\end{center}
Several of these involve a permanent expansion of our curriculum. The others are extra-curricular activities designed to complement what is learned in the classroom.  We have a track record in most of these areas, but we want to go far beyond our past experience to establish something substantial.


\subsection{\CampName} \label{Camp} For the past several years Illinois Tech's College of Science has run a three-week computational science summer camp for Chicago area high school students.  The goal has been to introduce them to solving mathematical and scientific problems using Mathematica.  The Admissions Office publicizes and recruits students for the camp.  Starting in the summer of 2018 this summer camp has an academic home in CISC, while continuing to partner with the Admissions Office.  Dr.~Kiah Wah Ong is the instructor.

We will broaden this camp to include computing on the CISC cluster to introduce these high school students to multi-core computing (Skills \ref{OneCPU}, \ref{TwoCPU}, and \ref{TwoCPUWrite}).  Mathematica has the capability to perform computations in parallel, and we will also investigate the use of other languages.  This summer camp will encourage the students to pursue degrees involving high performance computation and better prepare them for potential careers as computational scientists.  We also hope to attract the students to Illinois Tech's computational science offerings.  

Because our students are expected to have diverse backgrounds in computing, we will begin the camp by assessing their existing skills and the break them into groups so that those who need to learn more fundamental skills may do so, while those with more advanced knowledge can be given more challenging assignments.  Illinois Tech students will serve as teaching assistants (TAs) for the camp, which will provide them with the opportunity to reinforce their own learning by teaching the computational science that they have learned. 

Our goal is to host 20 high school students in the summer of 2019 and increase the number each year to 40 high school students steady state in the summer of 2021.  The summer camp is funded by tuition fees.  In order to broader access, we plan to offer tuition scholarships to those with financial need.  Due to the planned expansion, some grant funding may be required to support the tuition expenses in the short term.

\subsection{Strengthening undergraduate and graduate level computational science coursework} \label{Curr} 
Illinois Tech already has a substantial computational science curriculum.  However, it has several deficiencies, which we aim to address.

\subsubsection{\CurrExistName} \label{CurrExist} Illinois Tech quite a few several computational courses.  Computer science offers high performance computing at the undergraduate and graduate levels.  Applied mathematics BS through PhD students must all take at least one computational mathematics course and may choose from a handful of computational mathematics electives. There are two undergraduate computational physics courses, a computational chemistry course, and a computational biology course.

These courses teach Skills \ref{OneCPU}, \ref{EffOne}, and \ref{Natural}.  However, students tend to take only the courses in their own major.  Thus an applied mathematics student may not really have the chance to develop Skill \ref{Natural}.  The (co-)PIs, who teach some of these courses, will liaise  with other course instructors and the host departments to encourage non-majors to take these courses (Skill \ref{Other}).  This includes publicizing the relevant courses and exploring how we might lower the pre-requisite barriers.

\DMNote{This point doesn't seem very strong but it may be the best that we can do.}

\subsubsection{\RelSoftName} \label{RelSoft} In the Fall 2013 Choi and Hickernell piloted a one-credit course on Reliable Mathematical Software, primarily for their own graduate students.  The course covered truncation and round-off errors,
stopping criteria for adaptive software,
creating software collaboratively via Git, 
documentation,
input parsing and validation, and
reproducible computation.
Students completed a project consisting of a small software library or an extension to an existing library.  Several of the students in this course participated in developing GAIL) \cite{ChoEtal17b}.  This experience helped one student land an academic job teaching computer science, three students land computationally oriented positions in industry, and a fifth student gain admission to a PhD program.

In this project we will grow this to a standard three-credit undergraduate/graduate elective course.  In addition to a fuller treatment of the above topics, which address Skills \ref{Repro}--\ref{ContLib}, we will add material on working in a multi-lingual environment (Skill \ref{MultiLing}) and small-scale parallel computation (Skills \ref{TwoCPU} and \ref{TwoCPUWrite}).  The prerequisite will be any undergraduate course that covers numerical computation.

\subsubsection{\LargeSCName} \label{LargeSC} Since 2006 the computer science department has offered a graduate course, Advanced Scientific Computation, which covers Skills \ref{Tight}, \ref{AnalVis}, \ref{MultiLing}, \ref{HighLib}, and \ref{EffLarge}.  Dr.~Hong Zhang from Argonne National Laboratory has taught this course to students in computer science, mathematics, physics, and engineering. Students were exposed to cutting-edge analytic and algorithmic research projects and gained hands-on large-scale numerical programming experience. This course has been a launching point for many students into careers in computational science.  Several of the course projects were integrated into the Portable Extensible Toolobox for Scientific Computing (PETSc) \cite{petsc-web-page17}, benefiting the scientific community in large. At least eight students received summer internships at Argonne after taking her course, and three of her students eventually became post-doctoral researchers at Argonne.  

In the past, this course has run only every other year.  We plan to increase the frequency of this course to annually, serving a dozen or more students, by advertising its existence and benefits.  Students taking this course will be more competitive for the other opportunities below.

\subsubsection{\PhyChemBioCompSciName} \label{PhyChemBioCompSci} A new computational physics/chemistry/biology course using OpenScience Grid and XSEDE? \FJHNote{David and Jeff, something here for you?}
\DMNote{I could see either a computational natural science course or Open Science Grid/XSEDE usage, but combining both would be a bit tricky.}

\subsection{\CODSummerName} \label{CODSummer}

To  encourage the pursuit of bachelor's and advanced degrees by community college students, \JW has partnered in the past with the College of DuPage (COD) to recruit students from their associate degree programs  to engage in a ten-week summer internship program in his research group.  We propose to broaden this program to include several CISC associated research groups at Illinois Tech. Each year we will work with Prof.~Tom Carter of the physics department to advertise and recruit students to this program,  with a particular focus on underrepresented minorities.  These students will be matched with CISC labs based on their research interests and career goals.  Our goal is to aim for \DMNote{finish this sentence.}

In the first phase of their internships, the cohort of COD students will spend a week in an intensive computing ``crash course'' \DMNote{Perhaps mention that I will teach this ``taught by Minh''} that will include topics such as an introduction to linux, programming in python, and the use of high performance computing resources.  For the remaining nine weeks, CoD students will work directly in IIT labs on various computational projects, such as high performance Monte Carlo methods in Choi's and Hickernell's group, binding free energy calculations in the Minh's group, ??? in Sun's group \JWNote{help me list a few}, and simulations of biomolecular complexes in the Wereszczynski group.  To foster discussions among the CoD students, weekly lunches will be organized.  In addition, at the end of the summer a research symposium will be organized in which students will present a short presentation on their work.  Following the summer period, we will track the career trajectories of these students with annual follow up surveys, and we will maintain mentoring relationship with these students as are appropriate. 

We will aim for four COD students in 2019, six in 2020, and six in 2021.  Students will be provided \$5000 stipends.  After the project ends, we hope to secure funding to continue from other sources, such as Research Experience for Undergraduate (REU) funding and REU add-on supplemental funding.


\subsection{\FellowName} \label{Fellow}
To provide more opportunities for the practice of the advanced CI skills that we are teaching our students, we plan to offer summer fellowships to Illinois Tech undergraduate and graduate students.  These will be awarded on a competitive basis and be provided to students who partake one of the following experiences
\begin{itemize}
\item Embedding themselves in a computational science research group outside their major, or
\item Joining a large-scale computational science research project at Argonne, Fermilab, or in a local company.
\end{itemize}
\DMNote{What about adapting code for a specific natural science application from a single-core to a high-performance computing architecture? Would this fit under one of the above categories? If not could we consider adding it?}

The former opportunity moves our students outside their silos.  The broadening experience will make them more valuable members of interdisciplinary teams performing large-scale computations.  It will help them learn and practice Skills \ref{Natural}--\ref{Other}.  

CISC began a series of lunchtime matchmaking seminars in the fall of 2017 to introduce computational scientists and engineers at Illinois Tech to each others' research.  CISC also hosted a seed grant competition and awarded one grant to a multidisciplinary team.  The goal of these ongoing activities is to promote more competitive proposals for external funding of interdisciplinary computational science research.  This goal will be aided by the introduction of CISC fellows, while at the same time our other efforts to raise the awareness of computational science research at Illinois Tech should promote quality applications for the CISC Fellowships.

The ongoing large-scale computational projects at Argonne and Fermilab will provide students the opportunity to learn and practice Skills \ref{TopMach}, \ref{AnalVis}, \ref{MultiLing}, \ref{Decision}, and \ref{Model}.  Argonne has several software development projects including PETSc \cite{petsc-web-page17} and xSDK \cite{XSDK17a}. Fermilab has a heterogeneous computing environment for high energy physics (HEPCloud) \cite{HEP18a} under continuous development.  Two of our external advisory board members, Lois Curfman McInnes (Argonne) and Burt Holzman (Fermilab) will aid us in placing CISC Fellows in these two national labs.

The (co-)PIs have had collaborations and contacts with Chicago-area companies involved in supporting or performing large-scale calculations as part of the research.  We will further develop these contacts and identify sources of possible fellowships. \DMNote{It could help to name specific companies.}

Each year, starting in 2019, we will offer CISC fellowships to two undergraduates (\$5000 stipend each) and five graduate students (\$7000 stipend each).  The students will apply in the early part of the year, providing a synopsis of their proposed projects, their curriculum vitae, and letters of recommendation from their advisors and the project supervisors.  The management committee for this project---the (co-)PIs---will decide who will receive the fellowships.  

By the time that this grant is completed, we expect that our overall training program for computational scientists will have prepared a significant number of our to compete for funded internship opportunities in the national laboratories.  Moreover, we anticipate that embedding  students in research groups outside their silos will spark cross-disciplinary collaborations among research groups.  This in turn will lead to external funding of cross-disciplinary computational projects that will draw together and financially support computational science students from diverse majors.


\section{Broader Impacts}

\subsection{Educating the Next-Generation Computational Scientists} \DMNote{Remove ``the''?}
Standard science curricula have not kept pace with the potential of advanced CI for computational science.  Computer science courses teach the architectures and languages for large-scale computation, but they do not teach their students how to solve important scientific problems.  Computational mathematics, biology, chemistry, and physics courses do not educate their students in how best to use and contribute to advanced CI.

Our proposed computational science education program will set students on a path so that by the time they finish their PhD studies, they can be computational scientists making vital contributions to interdisciplinary research teams discovering new science.  Their work will lead to breakthroughs in our understanding nature, from the sub-atomic to the astrophysical scale.  Their work will lead to breakthroughs in health, medicine,  manufacturing and finance, which will improve the quality of our lives and the national economy. 

The students coming leaving our program will carry their academic and practical knowledge of advanced CI to their new research groups in universities, government labs, and companies that they will join.  They will share what they have learned with those who need to use more powerful computing capability, but do not yet know how.  The knowledge that we impart will be multiplied.

\subsection{Impacting Research in Computer Science, Mathematical Science, and Natural Science}
The interdisciplinary computational research promoted by our program will generate interesting research problems for the contributing disciplines:
\begin{itemize}
\item Complex quantitative models of natural science phenomena will challenge computational mathematicians and computer scientists to develop new algorithms that can utilize more advanced software and hardware.  
\item The advances in CI will alter how computational mathematicians and computer scientists design algorithms and measure algorithmic efficiency.  Efficiency measures for algorithms running on single processors do not apply to advanced CI, and new efficiency measures are needed.  Given the variety of architectures and languages, operation count or  clock time are not adequate good choices.
\item Large-scale scientific computation enabled by advanced CI will answer existing questions in the natural sciences and prompt new research questions in these sciences.  
\end{itemize}
The computational scientists educated by our program will be prepared to tackle these new research problems because they will be proficient contributors and users of advanced CI, i.e., CICs and CIUs.  They will know there own disciplines deeply enough to have the new ideas required to solve these new problems.  Because they are conversant in the other disciplines as well, they will know how to their ideas will help.

Consider an example involving quasi-Monte Carlo (qMC) methods (Hickernell's research area), which are used to evaluate multi-dimensional integrals. From their 1960s until the early 1990s the progress in qMC was mainly theoretical.  Then in the mid-1990s, Paskov and Traub \cite{PasTra95} showed that qMC could price a collateralized mortgage obligation much more efficiently than independent and identically distributed Monte Carlo (IIDMC).  This was surprising, since existing theory supporting showed that qMC was better than IIDMC up to dimension five or so, whereas Paskov and Traub's problem had dimension 360.  This computational result challenged mathematicians and theoretical computer scientists to re-visit the long-held theory, and the result was dozens of articles and a more complete understanding of the situations in which qMC is superior to MC, even for infinite dimensional integration.  New computation raised new research questions, which  were then answered.

As another example, a couple of years ago, Jim\'enez Rugama, the former PhD student of Hickernell, was invited to join a research group of physicists at Fermilab to explore whether their Monte Carlo (MC) computations could be sped up.  The default IIDMC algorithm being used was several decades old.  Jim\'enez Rugama was able to introduce the Fermilab group to the much more efficient qMC methods.  In the course of implementing these methods, Jim\'enez Rugama observed an anomaly in the calculations, which was traced to a bug in the long-standing physics code that generated the particle collision events.

\subsection{Modeling How Silos Can be Broken}
We recognize that computer science, mathematical science, and the several natural sciences are well-defined disciplines.  Someone living in one of these silos cannot take full advantage of advanced CI for scientific computation.  The solution is not to build a new computational science silo.  The solution is to develop computational scientists that are deep in one discipline but break out of their own silo to be conversant in others.

Several of the project personnel, including the PI, are conversant in languages and cultures other than the one that they were born into.  We recognize the the strength that comes not from homogenizing cultural differences but from drawing on the strengths of our birth culture while also understanding the cultures that are not our own.

As an analogy, the United States is comprised of people from a mixture of cultures.  Our strength comes not from homogenizing the cultural differences but from drawing on the strengths of our diverse cultural heritage while understanding the cultures that are not our own.  Most of the (co-)PIs have benefited by learning a language and culture other than the one they were born into.

This project will show other educational institutional how to break down the major discipline silos to foster high quality computational science.  The activities that we will implement will be models for others.  We will make our syllabi, training materials, and procedures available on repositories for others to follow and model. \DMNote{Maybe refer to data management plan for more specifics?} We will publicize our successes and the lessons learned from failures in conference talks.  We will advise those interested to adopt our best practices.

\subsection{Codifying Good Computational Science Practice}
The practice of good computational science is built upon diverse ideas that are not necessarily contained in one text or reference book.  Some texts on computational science, such as \cite{TveEtal10a}, focus on numerical methods, while others focus, such as \cite{Thi13a}, focus on on scientific applications.  Neither of these types of texts give significant attention to parallel computing or software engineering for high performance computational science software libraries. Texts on high performance computing for computational scientists, such as \cite{MagEtal16a}, are incomplete in their coverage of numerical methods.

As we develop the new course described in Sect.~\ref{RelSoft} and as we work with the CISC fellows, we will gather digestible resources describing good computational science practice. \DMNote{And then what? Write a review article? Put it on a web page?}

\subsection{Broadening Access to Computational Science Opportunities}
Illinois Tech's summer programs are aimed at serving the Chicago area high school students, especially those who are underrepresented minorities.  The goal is to provide intellectually stimulating experiences for them that will better prepare them for undergraduate and graduate study. 
In particular, our  summer course in computational science (see Sect.~\ref{Camp}) will broaden access to a career in computational science.  By teaching our students how compute answers to scientific problems \emph{and} giving them two college credits as well, we will be giving our students a head start that they would not have otherwise.

Community colleges enroll approximately 45\% of the nation's undergraduate students, a disproportionate number of which are underrepresented minorities \cite{KnappKG12,nsfreport13}.  Nearly 90\% of these students have a goal of transferring to four-year institutions to complete their bachelor's degree, however the actual transfer rate is estimated to be only 25-40\% \cite{HoachlanderSH03,MelguizoKA11}.  Studies have shown that involving students, especially those from underrepresented groups, in  research activities decreases  their attrition rate and increases the probability they will pursue further education \cite{BarlowV04,JonesBV10}.  Our research experience for COD undergraduates (Sect.~\ref{CODSummer}) will ignite their interest in computational science, giving them motivation to complete their undergraduate degrees and pursue advanced degrees.



\section{Partnerships and External Advisory Board} \label{PartnerSec}
To maximize the impact of our project, we are partnering with Ms.~April Welch, Associate Vice-President of Strategic Initiatives at Illinois Tech. Ms.~Welch and her colleagues in Illinois Tech's Office of Enrollment will assist us in recruiting students for the Summer Computational Science Course (see Sect.~\ref{Camp}).

We are also partnering with three other organizations through key contacts:
\begin{itemize}

\item Dr.~Tom Carter, Professor of Physics at College of DuPage (COD), a community college in the western suburbs of Chicago, 

\item Dr.~Lois Curfman McInnes, Senior Computational Scientist in the Mathematics and Computer Science Division at Argonne National Laboratory, where she was former PETSc co-lead and is presently xSDK co-lead, and

\item Dr.~Burt Holzman, Assistant Director of the Scientific Computing
Division at Fermi National Accelerator Laboratory (Fermilab), where he oversees the
HEPCloud program and coordinates cross-cutting initiatives and solutions
across the facility.

\end{itemize}


Dr.~Carter will partner with us to identify COD students most suited to take advantage of our summer research experience for under-served undergraduates (see Sect.~\ref{CODSummer}).  He will help communicate to the students our expectations for the summer and feedback to us any difficulties that the students are having.  He will also let us know these students' next steps after finishing their COD studies.

Dr.~Curfman McInnes will help us identify large-scale computation projects and advisors at Argonne that are suitable for CISC Fellows (see Sect.~\ref{Fellow}).  Particular attention will be given to PETSc and xSDK related projects.  She will advise in the selection of the CISC Fellows to be placed at Argonne.

Dr.~Holzman will help us identify large-scale computation projects and advisors at Fermilab that are suitable for CISC Fellows (see Sect.~\ref{Fellow}).  Particular attention will be given to HEPcloud projects involving heterogeneous systems.  He will advise in the selection of the CISC Fellows to be placed at Fermilab.

Our External Advisory Board will be comprised of Dr.~Carter, Dr.~Curfman McInnes, and Dr.~Holzman.  This board will meet with the (co-)PIs and Senior Personnel twice a year (by teleconference if necessary), to review the progress of our project and advise us on how our efforts can be even more effective.  Apart from these formal meetings, we will welcome their advice at any time and solicit their advice as needed.

\section{Evaluation}



\newpage \setcounter{page}{1} %%%%%%%%%%%%%%%%%%%%%%%%%%%%%%%%%%%%%%%%%%%%%%%%%%



\bibliographystyle{spbasic}

{\renewcommand\addcontentsline[3]{} 
\renewcommand{\refname}{{\Large\textbf{References Cited}}}                   %%
\renewcommand{\bibliofont}{\normalsize}

\bibliography{FJH23,FJHown23,GregPapers,jeff}}

\newpage \setcounter{page}{1} %%%%%%%%%%%%%%%%%%%%%%%%%%%%%%%%%%%%%%%%%%%%%%%%%%

\centerline{\textbf{\Large Management and Coordination Plan}}
% Supplementary Document 1. Management and Coordination Plan (2 pages): Each proposal must contain a clearly-labeled Management and Coordination Plan that includes: 1) the specific roles of the PI, co-PIs, other Senior Personnel and paid consultants at all institutions involved; 2) how the project will be managed across institutions and disciplines; 3) identification of the specific coordination mechanisms; and 4) pointers to the budget line items that support these management and coordination mechanisms.



\bigskip

The (co-)PIs, all of whom are from Illinois Tech, will serve together as the Management Team for this project:
\begin{itemize}
\item Fred J. Hickernell (PI), Professor of Applied Mathematics and Director of the Center for Interdisciplinary Scientific Computation (CISC), 
\item Sou-Cheng Choi (co-PI), Research Associate Professor of Applied Mathematics and Lead Researcher at Allstate,
\item David Minh, Assistant Professor of Chemistry and Associate Director of CISC,
\item Xian-He Sun, Distinguished Professor of Computer Science, and 
\item Jeff \JW, Assistant Professor of Physics. 
\end{itemize}
The Management Team will meet bi-monthly to share progress of the project initiatives, discuss challenges that arise, and strategize on how to make our initiatives more effective.

The Management Team will be assisted by the following Illinois Tech colleagues:
\begin{itemize}
\item Norman Lederman (Senior Personnel), Distinguished Professor of Science Education,
\item Kiaw Wah Ong  (Senior Personnel), Lecturer in Mathematics, and 
\item April Welch (Internal Collaborator), Associate Vice-President of Strategic Initiatives. \end{itemize}
They will meet with the whole Management Team or with individual (co-)PIs as needed.

As mentioned in Sect.\ \ref{PartnerSec} of the Project Description, an External Advisory Board will be formed, consisting of the following members:
\begin{itemize}
\item Dr.~Tom Carter, Professor of Physics at College of DuPage (COD),

\item Dr.~Lois Curfman McInnes, Senior Computational Scientist in the Mathematics and Computer Science Division at Argonne National Laboratory, and

\item Dr.~Burt Holzman, Assistant Director of the Scientific Computing
Division at Fermi National Accelerator Laboratory (Fermilab).

\end{itemize}

\newcommand{\Salaries}{Sal}
\newcommand{\Stipends}{Sti}
\newcommand{\Travel}{Tra}




The table below shows the persons responsible for each major task.  The first person listed for each task takes the lead.  The budget column lists the financial support for each item, including summer salaries (\Salaries), stipends (\Stipends), and travel (\Travel).  HSal denotes salary for Hickernell, with M\Salaries, S\Salaries, and W\Salaries having analogous meanings.

\bigskip
\begin{longtable}
{r>{\raggedright}p{0.52\textwidth}>{\raggedright}p{0.25\textwidth}>{\raggedright}p{0.15\textwidth}}
\multicolumn{2}{>{\raggedright}p{0.55\textwidth}}{\textbf{Task} (with reference to the Project Description)} & \textbf{Persons} &\textbf{Budget} \tabularnewline
\toprule
\ref{Camp} & \CampName \tabularnewline
& \itemdash Course design and content creation & Ong, Hickernell & H\Salaries\tabularnewline
& \itemdash Advertisement to prospective students & Welch, Ong \tabularnewline
& \itemdash Instruction and supervision of TAs & Ong & tuition \tabularnewline
& \itemdash Evaluation & Lederman, Ong & L\Salaries\tabularnewline
\ref{CurrExist} & \CurrExistName\tabularnewline
& \itemdash Applied mathematics & Hickernell & H\Salaries\tabularnewline
& \itemdash Chemistry & Minh & M\Salaries \tabularnewline
& \itemdash Computer science & Sun & S\Salaries\tabularnewline
& \itemdash Physics & \JW & W\Salaries \tabularnewline
\ref{RelSoft} &\RelSoftName \tabularnewline
& \itemdash Course design and content creation & Hickernell, Choi \tabularnewline
\ref{LargeSC} & \LargeSCName & Sun \tabularnewline
\ref{CODSummer} & \CODSummerName & \JW, Minh, Carter  &W\Salaries, M\Salaries, \Stipends \tabularnewline
& \itemdash Identifying students from COD & \JW, Carter & W\Salaries\tabularnewline
& \itemdash Crash preparatory course design and instruction & Minh, \JW & M\Salaries, W\Salaries \tabularnewline
\ref{Fellow} & \FellowName & Hickernell, Choi,   \tabularnewline
Organizing meetings of the Management Team and the External Advisory Board & Hickernell &H\Salaries \tabularnewline
\end{longtable}




\begin{longtable}
{>{\raggedright}p{0.55\textwidth}>{\raggedright}p{0.25\textwidth}>{\raggedright}p{0.15\textwidth}}
\textbf{Task} (with reference to the Project Description) & \textbf{Persons} &\textbf{Budget} \tabularnewline
\toprule
\ref{Camp}. \CampName \tabularnewline
\quad \itemdash Course design and content creation & Ong, Hickernell & H\Salaries\tabularnewline
\quad \itemdash Advertisement to prospective students & Welch, Ong \tabularnewline
\quad \itemdash Instruction and supervision of TAs & Ong & tuition \tabularnewline
\quad \itemdash Evaluation & Lederman, Ong & L\Salaries\tabularnewline[-2ex]
\begin{tabbing} \ref{CurrExist} \CurrExistName \end{tabbing}
\tabularnewline[-3ex]
\quad \itemdash Applied mathematics & Hickernell & H\Salaries\tabularnewline
\quad \itemdash Chemistry & Minh & M\Salaries \tabularnewline
\quad \itemdash Computer science & Sun & S\Salaries\tabularnewline
\quad \itemdash Physics & \JW & W\Salaries \tabularnewline
\ref{RelSoft} \RelSoftName \tabularnewline
\quad \itemdash Course design and content creation & Hickernell, Choi \tabularnewline
\ref{LargeSC} \LargeSCName & Sun \tabularnewline
\ref{CODSummer}. \CODSummerName & \JW, Minh, Carter  &W\Salaries, M\Salaries, \Stipends \tabularnewline
\quad \itemdash Identifying students from COD & \JW, Carter & W\Salaries\tabularnewline
\quad \itemdash Crash preparatory course design and instruction & Minh, \JW & M\Salaries, W\Salaries \tabularnewline
\ref{Fellow}. \FellowName & Hickernell, Choi,   \tabularnewline
Organizing meetings of the Management Team and the External Advisory Board & Hickernell &H\Salaries \tabularnewline
\end{longtable}


PI meeting at the NSF

\subsection*{Timeline}

\newpage \setcounter{page}{1} %%%%%%%%%%%%%%%%%%%%%%%%%%%%%%%%%%%%%%%%%%%%%%%%%%

\centerline{\textbf{\Large Data Management Plan}}

\bigskip



This plan will make certain that the data produced during the period of this project is appropriately managed to ensure its usability, access and preservation.  The data produced by the proposed project will consist of cyberinfrastructure theory, new software, good practices for cyberinfrastructure training, course materials, program guidelines. 

\subsection*{Peer-Reviewed Articles}  The PI, Co-PI, and participants in this project will publish the results of their theoretical, and computational investigations as early as appropriate in the form of peer-reviewed journal articles, conference abstracts and talks at various conferences and institutions. Authorship will accurately 
reflect the contributions of those involved.  Students will be particularly encouraged to publish their work. When allowed by publishers, pre-prints of publications will be posted on arXiv.

\subsection*{Contributions to Computational Science Forums} The best practices that we discover and the resources that we develop during this project will be contributed 

\subsection*{Software}
Most software resulting from this project will fall within the scope of the Guaranteed Automatic 
Integration Library (GAIL) and will become part of future public releases.  GAIL will be hosted on 
Github or a similar accessible repository as it is now.  It will be publicized through colloquium and 
conference talks and e-newsletters such as the NA-Digest.

Substantial software resulting from this project that does not fit within GAIL will likewise be 
published in an accessible repository.

In the spirit of reproducible research, the code used for numerical tests and figures for our 
publications will be included in the GAIL and other repositories.

\subsection*{Course Materials}

\subsection*{Program Practices and Guidelines}\ 

We will fully comply with all applicable guidelines and policies on model and data sharing as mandated or recommended by NSF.

This Data Management Plan addresses NSF’s policy on the dissemination and sharing of research results within a reasonable time.  In accordance with this policy, this plan does not include preliminary analyses (including raw data), drafts of scientific papers, plans for future research, peer reviews, or communications with colleagues. 


\newpage \setcounter{page}{1} %%%%%%%%%%%%%%%%%%%%%%%%%%%%%%%%%%%%%%%%%%%%%%%%%%


\centerline{\textbf{\Large Facilities, Equipment and Other Resources}}

\bigskip

Sou-Cheng Terrya Choi, the co-PI will join the regular management team meetings, in addition to working remotely.  She is employed full-time outside academia, but contributes to the project on a volunteer 
basis.  No salary is requested for her, but her travel related to the project will be supported.

All Illinois Tech faculty, PhD students, and visitors have offices provided at Illinois Tech.  Summer MS, BS, and high school students have shared work areas.  In addition to faculty, student and visitor 
offices and conference rooms provided by the Department of Applied Mathematics, the Center for Interdisciplinary Scientific Computation (CISC) also has office and meeting space available for use by this project.

CISC has a 256-core cluster named von Neumann funded by the 
College of Science.  An increase in the number of cores within the year is likely. Von Neumann is available available to all Illinois Tech research faculty and is
centrally managed by Illinois Tech Office of Technology (OTS) Services.  Illinois Tech is connected to the \href{https://www.opensciencegrid.org}{Open Science Grid} through its own GridIIT.  

\FJHNote{XHS, please add your resources here}

Illinois Tech has site licenses for Mathematica, MATLAB, SAS, and JMP.  Other open source software is also installed in our research and teaching laboratories.

Illinois Tech's university library provides access to journals, research monographs, and databases, either on-site, online or via inter-library loan.

Illinois Tech was listed on the National Federal Register of Historic Places in 2005. The proposed research activities will not make any physical changes to Illinois Tech's campus and buildings.


\newpage \setcounter{page}{1} %%%%%%%%%%%%%%%%%%%%%%%%%%%%%%%%%%%%%%%%%%%%%%%%%%

\centerline{\textbf{\Large Budget Justification}}

\subsection*{Salaries and Wages}

Fred J. Hickernell, the PI is budgeted at two months of summer salary per year.  He will lead 
the research.  This includes leading the weekly research seminar and  meeting individually with the 
students and visitors involved in this project.

The proposed salary for the graduate research assistant is vital, not only for the training of students, 
but also as brainpower to contribute to the construction, analysis, and implementation of our new 
algorithms.  The graduate research assistant will also help to mentor the REU students.

The undergraduate salaries are for two REU students per summer.  This research experience helps 
prepare students for further study and STEM careers by exposing them to the solving of problems 
whose answers are yet unknown.

\subsection*{Fringe Benefits}
Fringe benefit rate is 23.8\% for academic year salary and 7.9\% for the summer month
salary.

\subsection*{Travel}
The PI and co-PI will travel to several conferences each year to disseminate their work and to learn 
what others have discovered.  These include, e.g., annual meetings of the major mathematical and 
statistical societies, specialized meetings in Monte Carlo methods and related fields, and regional 
conferences. 

We often take our students to conferences with us so that they may have the experience of 
presenting their work and networking with the academic community.  The REU students are always 
encouraged to attend conferences to present their work. 

\subsection*{Other Direct Costs - Tuition}
The current tuition rate is \$1,470 per credit hour. Amount requested is 
\$1,470 $\times$ 9 credit hours per year = \$13,200 per year.

\subsection*{Indirect Costs}
Our federally negotiated rate is 53\% of modified total direct costs.

\subsection*{Inflation}A 4\% annual increase applies to all categories


\end{document}

\newcommand{\CampName}{Summer camp for high school students}
\newcommand{\CurrExistName}{Increasing accessibility to existing computational science offerings}
\newcommand{\RelSoftName}{Professional practices for computational science course} 
\newcommand{\LargeSCName}{Large-scale scientific computation course}
\newcommand{\PhyChemBioCompSciName}{New computational natural sciences course}
\newcommand{\CODSummerName}{Research experiences for under-served undergraduates}
\newcommand{\FellowName}

1. Management and Coordination Plan (2 pages): Each proposal must contain a clearly-labeled Management and Coordination Plan that includes: 1) the specific roles of the PI, co-PIs, other Senior Personnel and paid consultants at all institutions involved; 2) how the project will be managed across institutions and disciplines; 3) identification of the specific coordination mechanisms; and 4) pointers to the budget line items that support these management and coordination mechanisms.



\end{document}


