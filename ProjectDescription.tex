\documentclass[11pt]{NSFamsart}
\usepackage[utf8]{inputenc}
\usepackage{xspace}
\usepackage[dvipsnames]{xcolor}
\usepackage[numbers]{natbib}
\usepackage{hyperref,array,accents,longtable,booktabs}
\usepackage{cleveref}
%\usepackage[notref,notcite]{showkeys} %This package prints the labels in the margin

\setlength{\leftmargini}{2.5ex} %indentation of the left margin of the itemize


\thispagestyle{plain}
\pagestyle{plain}

\headsep-0.6in
\textwidth6.5in
\oddsidemargin0in
\evensidemargin0in
\textheight9in

\newcommand{\myshade}{85}
\colorlet{mylinkcolor}{violet}
\colorlet{mycitecolor}{Aquamarine}
\colorlet{myurlcolor}{YellowOrange}
\hypersetup{ %make the links stand out
	linkcolor  = mylinkcolor!\myshade!black,
	citecolor  = mycitecolor!\myshade!black,
	urlcolor   = myurlcolor!\myshade!black,
	colorlinks = true,
}

\providecommand{\FJHickernell}{Hickernell}

% Everyone feel free to add their own note definitions here
\newcommand{\FJHNote}[1]{{\textcolor{blue}{FJH: #1}}}
\newcommand{\DMNote}[1]{{\textcolor{green}{DM: #1}}}
\newcommand{\JWNote}[1]{{\textcolor{orange}{JW: #1}}}
\newcommand{\SCCNote}[1]{{\textcolor{magenta}{SCC: #1}}}
\newcommand{\XHSNote}[1]{{\textcolor{red}{XHS: #1}}}
\newcommand{\NLNote}[1]{{\textcolor{yellow}{NL: #1}}}
\newcommand{\KWONote}[1]{{\textcolor{cyan}{KWO: #1}}}



\newcommand{\JW}{Wereszczynski\xspace} %help me spell his name correctly
\newcommand{\Order}{\mathcal{O}}







\newcounter{skillct}
\newcommand{\skillnum}[1]{\refstepcounter{skillct}\label{#1}\arabic{skillct}.}
\newcommand{\notyet}{\textbf{?}}
\newcommand{\done}{\checkmark}




\begin{document}
\leftmargini2.5ex %indentation of the left margin of the itemize

\centerline{\Large \textbf{To Do List}}
\bigskip

\begin{itemize}
\item[\done] Confirm Title: CyberTraining: CIC:  Cross-Disciplinary Education for Next-Generation Computational Scientists
\item Project Description
\begin{itemize}
\item Previous funding
\begin{itemize}
\item[\done] Fred
\item[\notyet] Xian-He
\item[\notyet] David
\item[\done] Jeff
\item[\done] Sou-Cheng
\item[\notyet] Norm
\item[\notyet] Kiah Wah
\end{itemize}
\item[\done] References (Jeff)
\item[\notyet] Graduate course in computational natural science (David, Jeff)
\item[\notyet] Broader impacts
\item[\notyet] Evaluation (Norm)
\end{itemize}

\item[\notyet] Budget

\item NSF Biosketches
\begin{itemize}
\item[\notyet] Fred
\item[\done] Sou-Cheng
\item[\done] Xian-He
\item[\notyet] David
\item[\notyet] Jeff
\item[\notyet] Norm
\item[\notyet] Kiah Wah
\end{itemize}


\item Collaborators and other affiliations
\begin{itemize}
\item[\notyet] Fred
\item[\notyet] Sou-Cheng \SCCNote{Sent to Brian Davis  at \texttt{jdavis10@iit.edu} on Feb 1, 2018. A copy, \texttt{20180128\_Sou\_Cheng\_Choi\_COA.xlsx}, is uploaded to Overleaf's project folder \texttt{Biosketch}.}
\item[\notyet] Xian-He
\item[\notyet] David
\item[\notyet] Norm
\item[\notyet] Kiah Wah
\end{itemize}

\item Facilities
\begin{itemize}
\item[\notyet] Xian-He's lab
\end{itemize}

\item Letters of Collaboration
\begin{itemize}
\item[\notyet] Lois
\item[\notyet] Burt
\item[\notyet] COD
\item[\notyet] April Welch
\end{itemize}

\item[\notyet] Data Management

\item[\notyet] Management
\end{itemize}

\bigskip

\noindent
\XHSNote{A note from Xian-He}\\
\DMNote{A note from David} \\
\JWNote{A note from Jeff} \\
\SCCNote{A note from Sou-Cheng} \\
\NLNote{A note from Norm Lederman}\\
\KWONote{A note from KiahWahOng}\\

\newpage
\setcounter{page}{1}

\centerline{\Large \textbf{Project Description}}
\vspace{-2ex}

\setcounter{tocdepth}{2}
\tableofcontents %Help the readers navigate the proposal

\vspace{-6ex}

\section{Introduction}
Computational sciences typically draw upon the knowledge and strengths of several disciplines, including computer science, mathematical science, and natural sciences.  Realizing the full potential of computational science for discovery requires multidisciplinary teams whose members can take full advantage of advanced hardware architectures and software environments.  Computational scientists often require the following qualities to conduct effective research:

\begin{itemize}
\item depth within their chosen disciplines plus breadth across relevant disciplines, overcoming a silo mentality, and
\item experience with advanced cyberinfrastructure (CI), including the practices that extend beyond what is required for computations on a single processor or a core  in a multi-core CPU (central processing unit).
\end{itemize}

The newly established Center for Interdisciplinary Scientific Computation (CISC) \url{http://cos.iit.edu/cisc} at Illinois Institute of Technology (IIT, or Illinois Tech) will lead the development of a new program to educate computational scientists that have the two attributes above.  We will educate high school through doctoral students.  Learning will be curricular and extra-curricular.  We will partner with nearby national laboratories, companies engaged in advanced computing, and schools whose students have less access to research experience and high-end computing facilities.

\subsection*{The Challenges}
Computing solutions to complex scientific problems requires properly educated computational scientists.  This proper education must overcome two kinds of challenges.

\subsubsection*{Academic Silos.} Computational science draws on multiple disciplines, including computer science, mathematical science, and natural sciences.  Students must gain a deep understanding of their chosen discipline. Computational scientists should not be a jack of all trades and a master of none.

\begin{itemize}
\item Computer scientists must understand how emerging languages and architectures enable faster and more scalable computations.  

\item Computational mathematicians must be able to describe and analyze truncation and round-off errors in numerical algorithms, as well as measurement and misspecification errors of statistical models.  They must know how to choose the most efficient algorithms for different kinds of problems.

\item Computational biologists, chemists, and physicists must grasp the important scientific ideas that need to be captured by computations.  They must understand the benefits and limitations of using computer modeling in their respective disciplines.

\end{itemize}
But this is not enough!  Computational scientists should be a master of one trade and conversant with others.  To be effective members of interdisciplinary teams tackling large, complex computational science problems, computational scientists must be ready to exit their silos and become familiar with other disciplines.

\subsubsection*{Ignorance of Good Practice for Large Scale Computation.}  Solving large, complex computational problems using advanced CI deviates in important ways from what students typically learn in their coursework. See \cite[Sect.\ 2.4]{RudEtal18a} for a more detailed discussion.  Here are some highlights:

\begin{itemize}

\item Effective algorithms that take advantage of muti-core, distributed memory architectures may be substantially different than algorithms designed for a single CPU with only one core.

\item The software required to solve large problems is often drawn from multiple sources, written in several languages, and developed over the course of years, by different experts.

\item Those who perform computational experiments and those who demonstrate the performance of their new algorithms must ensure that their results can be \emph{reproduced} by others \cite{Pen11}.  

\item Software contributors must ensure that their software is robust, interacts well with other software, and can be extended by those who come afterwards.  See \cite{BSS18} for a recent effort to promote these principles broadly.

\end{itemize}

We propose to overcome these challenges in educating the next generation of computational scientists.

\subsection*{Why CISC is Poised to Lead This}
CISC was created in May 2017 to leverage Illinois Tech’s existing strengths in computational science for greater impact.  CISC has been given office space, a modest budget, a 256-core cluster, and the good will of supporters among faculty and administrators.  Since its inception, CISC has initiated a series of lunchtime matchmaking seminars \SCCNote{matchmaking means?}, brought its cluster online, hosted special lectures, and sponsored a seed grant competition. Fred J. Hickernell, PI and CISC’s director, is one of several computational mathematicians at Illinois Tech and previously served for twelve years as chair of the applied (and only) mathematics department.  Our computer science department has a historically strong group in high performance computing led by Xian-He Sun, a co-PI.  In recent years, the biology, chemistry, and physics departments have hired computational scientists, including co-PIs David Minh (chemistry, CISC's associate director) and Jeff Wereszczynski (physics).  Co-PI Sou-Cheng Choi is a lead researcher at Allstate Insurance Company's Automotive and Life/Retirement innovation teams, and also a research associate professor in applied mathematics at Illinois Tech.

The (co-)PIs have experience developing curricula, mentoring high school students through PhD students, and partnering with College of DuPage, Argonne National Laboratory, and Fermilab.  Our proposed innovations will build upon our track record to build a strong multi-faceted training program for computational scientists.

\section{Results from Prior NSF Support} \FJHNote{Everyone needs to fill out their previous experience, need subsections on intellectual merit and broader impacts}
\subsection{Experience of Hickernell and Choi}
NSF-DMS-1522687\except{toc}{, \emph{Stable, Efficient, Adaptive Algorithms for Approximation and 
Integration}, \$270,000, August 2015 -- July 2018.} \label{SectHickernellPrevious}  Hickernell is PI and Choi is senior personnel.  

\subsubsection{Intellectual Merit}
One of the primary outcomes of this project in relation to the present proposal is the development of the Guaranteed Automatic Integration Library (GAIL) \cite{ChoEtal17b}.  This library comprises univariate and multivariate integration, univariate function approximation, and univariate optimization algorithms that automatically determine the sample size required to meet user-defined error tolerances.  The library does not rely on interval arithmetic, as is done in INTLAB \cite{MoKeCl09, Rum99a, Rum10a}, but like INTLAB, GAIL comes with theoretical guarantees that common adaptive algorithms lack. The recent GAIL developments include locally adaptive function approximation and optimization \cite{ChoEtal17a, Din15a}, adaptive quasi-Monte Carlo cubature \cite{HicJim16a, JimHic16a}, and the ability to set a hybrid error tolerance involving both absolute and relative error criteria~\cite{HicEtal17a}.  Other Articles, theses,  software, and preprints supported in part by this grant include 
\cite{ala_augmented_2017, 
	GilEtal16a,
	GilJim16b,
	HicEtal18a,	
	Hic17a,
	JohFasHic18a,
	Li16a,
	Liu17a,
	mccourt_stable_2017,
	mishra_hybrid_nodate,
	mishra_stable_nodate, 
	rashidinia_stable_nodate,
	vu_rbf-fd_nodate,
	Zha17a,
	Zho15a,
	ZhoHic15a}.
    
\subsubsection{Broader Impacts}  Three students (two female) have completed their PhD degrees, and three PhD students are in the midst of their PhD degrees.  One PhD graduate spent time working at Fermilab as a student helping them implement modern Monte Carlo methods.  Another PhD student is picking up where the first one left off.  Two students (one female) completed their MS theses, and one female student has nearly completed her MS thesis. More than a dozen undergraduate students have been mentored (primarily during the summer),  over the course of three summers.  Hickernell has embedded the new adaptive (quasi-)Monte Carlo research in his yearly graduate Monte Carlo class, and some students in that class have contributed to GAIL.  Hickernell, Choi, and their students have written an encyclopedia article, given numerous conference and colloquium talks, and organized a conference and special conference sessions at multiple conferences.  Hickernell has given an invited conference tutorial and is one of the program leaders for this year's SAMSI program on quasi-Monte Carlo sampling. Hickernell received the 2016 Joseph F.\ Traub Prize for Achievement in Information-Based Complexity.




\subsection{Experience of Sun}

\subsection{Experience of Minh}

DM is a new investigator and has not previously received NSF funding.  

\DMNote{Fred, should I still say something about my research (unfunded and NIH funded) and broader impacts?}

\FJHNote{David, yes, please say something.}

\subsection{Experience of Wereszczynski}

%\JWNote{I have two NSF grants to report on.  Note sure how you want to arrange them}

PI of NSF-MCB-1552743   ``CAREER: The Effects of Post-translational Modifications and Histone Variants on Chromatin Fiber Dynamics'' June 2016 -- May 2021, \$790,129.


\subsubsection{Intellectual Merit} The aim of this CAREER award is to use and develop multiscale biophysical simulation techniques to study how DNA is compacted in the cell, and how these mechanisms are regulated to affect gene expression.  The PI's group addresses these issues through the use of molecular dynamics (MD) simulations, along with collaborations at Argonne National Laboratory and the University of Iowa.  Early results have been presented at the annual Biophysical Society Meeting in 2017 and 2018, and a manuscript detailing the mechanisms by which post-translational modifications to the ``H3 tails'' affect the structures and energetics of nucleosomes is under minor review at \textit{eLife.} \SCCNote{Add a citation?}

\subsubsection{Broader Impacts} To date, this proposal has funded four trainees: one postdoctoral scholar, one PhD student, and two undergraduates from the College of DuPage (COD).  Both undergraduates performed 10-week summer research projects, during which time they learned how to set up, perform, and analyze MD simulations of various biomolecular systems. Mr.~Robert Hickok, who was in our lab in the summer of 2016, is currently enrolled at the University of Illinois at Chicago, whereas Mr.~Meet Patel, who worked with us in 2017, is enrolled at Georgia Institute of Technology.  


%\subsubsection*{NSF-MCB-1716099
%Collaborative Research: Molecular Mechanism of Heme Extraction by IsdH''   August 2017-July 2020 \$250,000 (collaborative with Robert T. Clubb, UCLA, \$700,000 total budget)}
%\textbf{Intellectual Merit}  This is a newly awarded proposal that supports an ongoing collaboration between our group and the Clubb group at UCLA.  The aim of this proposal is to use experimental and computational techniques to understand the mechanisms by which the bacterial protein IsdH scavenges iron from host hemoglobin.  Initial results of our modeling results will be presented at the 2018 Biophysical Society Meeting and a join experimental and computational manuscript is currently under minor revision publication in the Journal of Biological Chemistry.

%\textbf{Broader Impacts} To date, only one trainee has been supported by this proposal: Joseph Clayton, a PhD student in our group.  It is expected that in the summer of 2018 another undergraduate student from CoD will be supported on this award.

\section{Intellectual Merit}

\subsection{Skills to Be Learned}
Our primary emphasis is educating CI contributors (CICs), and our secondary emphasis is educating CI users (CIUs).  Our goal is to train students from high school through PhD level in skills of various difficulty, some building upon others.  These include the following.

\begin{longtable}
{r>{\raggedright}p{0.62\textwidth}ccc}
\multicolumn{2}{l}{Learning Outcomes} & CIU & CIC & Learned\\
\toprule
\skillnum{OneCPU} & Write and run numerical programs on a single CPU &
HS & HS & \ref{Camp}, \ref{CurrExist}, \ref{CODSummer} \tabularnewline
\skillnum{TwoCPU} & Write and run numerical programs that take advantage of multiple cores and/or a GPU on a single machine &
UG & HS & \ref{Camp}\tabularnewline
\skillnum{MultiJobs} & Run multiple jobs simultaneously on a cluster &
UG & HS & \ref{Camp} \tabularnewline
\skillnum{Tight} & Run jobs that require multiple processors with tight connectivity &
G & UG & \ref{Fellow} \tabularnewline
\skillnum{TopMach} & Run jobs on a top-500 machine &
G & UG & \ref{Fellow} \tabularnewline
\skillnum{AnalVis} & Use tools for analysis and visualization of large scale simulations &
G & G & \ref{LargeSC}, \ref{Fellow} \tabularnewline
\skillnum{MultiLing} & Solve scientific problems that require multiple libraries and languages & G & UG & \ref{Fellow} \tabularnewline
\skillnum{Repro} & Execute reproducible scientific computations & UG & UG & \ref{RelSoft} \tabularnewline
\skillnum{ContLib} & Contribute to a well-documented numerical software library consisting of documented, tested, robust routines &
UG & UG & \ref{RelSoft} \tabularnewline
\skillnum{HighLib} & Contribute to a numerical software library takes advantage of high performance computing architectures &
 & G & \ref{Fellow} \tabularnewline
\skillnum{EffOne} & Analyze the computational efficiency of individual algorithms and identify performance bottlenecks &
& UG & \ref{CurrExist} \tabularnewline
\skillnum{EffLarge} & Analyze the computational efficiency of large scale simulations and identify performance bottlenecks &
& G & \ref{Fellow}\tabularnewline
\skillnum{Natural} & Evaluate whether simulation output accurately reflects the natural phenomenon it is designed to emulate &
UG & UG & \ref{CurrExist}, \ref{Fellow} \tabularnewline
\skillnum{Decision} & Appreciate how a computation informs decision-making in the application domain and its effect on the kind of computation needed & G & G & \ref{Fellow} \tabularnewline
\skillnum{Model} & Know how different modeling assumptions in the application domain are tied to the choice of different computational methods and computing environments & G & G & \ref{Fellow} \tabularnewline
\skillnum{Other} & Collaborate with computational scientists outside their own majors & UG & UG & \ref{Fellow} \tabularnewline
& \FJHNote{What are we missing?  Please add} \tabularnewline
\bottomrule
\end{longtable} 

Many of the learning outcomes above are parallel to those given in \cite[Table 1]{RudEtal18a}.  However, that table focuses on the outcome for a PhD in computational science.  Our table above provides intermediate outcomes for those in high school, bachelor's, and master's study.  A future CIC should ideally develop all of the skills labeled HS in high school, all of the skills labeled UG during undergraduate studies, and all of the skills labeled G during graduate studies.  The analogous interpretation applies to CIUs.  Our proposed curricular and extra-curricular innovations, outlined in the remainder of this section, provide opportunities to future CICs and CIUs to develop these skills.  

The degree of mastery of any particular skill may depend on whether the computational scientist is majoring in computer science, mathematical science, or natural science.  Moreover, our students may only be with us for part of their high school through PhD studies.  We are ready to help those whose preparation is lacking to catch up.  When students leave our program they should be fully prepared for the next steps in their growth as computational scientists.

\subsection{Program Innovations}
\newcommand{\CampName}{Summer camp for high school students}
\newcommand{\CurrExistName}{Increasing accessibility to existing computational science offerings}
\newcommand{\RelSoftName}{Professional practices for computational science course} 
\newcommand{\LargeSCName}{Large-scale scientific computation course}
\newcommand{\PhyChemBioCompSciName}{New computational natural sciences course}
\newcommand{\CODSummerName}{Research experiences for under-served undergraduates}
\newcommand{\FellowName}{CISC summer fellowships}

Our innovations will better prepare CICs and CIUs via several modes of learning for different kinds of students.  They can be summarized as follows:

\begin{center}
\begin{tabular}
{lcc}
Innovation & \multicolumn{2}{c}{Target} \tabularnewline
\toprule
\CampName & CIU & CIC \tabularnewline
\CurrExistName & CIU & CIC \tabularnewline
\RelSoftName &  & CIC \tabularnewline
\LargeSCName &  & CIC \tabularnewline
\PhyChemBioCompSciName & CIU & CIC \tabularnewline
\CODSummerName & CIU & CIC \tabularnewline
\FellowName &  & CIC \tabularnewline
\bottomrule
\end{tabular}
\end{center}
Several of these involve a permanent expansion of our curriculum. The others are extra-curricular activities designed to complement what is learned in the classroom.  We have a track record in most of these areas, but we want to go far beyond our past experience to establish something substantial.


\subsection{\CampName} \label{Camp} For the past several years Illinois Tech's College of Science has run a three-week computational science summer camp for Chicago area high school students.  The goal has been to introduce them to solving mathematical and scientific problems using Mathematica.  The Admissions Office publicizes and recruits students for the camp.  Starting in the summer of 2018 this summer camp has an academic home in CISC, while continuing to partner with the Admissions Office.  Dr.\ Kiah Wah Ong is the instructor.

We will broaden this camp to include computing on the CISC cluster to introduce these high school students to multi-core computing (Skills \ref{OneCPU}--\ref{MultiJobs}).  Mathematica has the capability to perform computations in parallel, and we will also investigate the use of other languages.  This summer camp will encourage the students to pursue degrees involving high performance computation and better prepare them for potential careers as computational scientists.  We also hope to attract the students to Illinois Tech's computational science offerings.  

Because our students are expected to have diverse backgrounds in computing, we will begin the camp by assessing their existing skills and the break them into groups so that those who need to learn more fundamental skills may do so, while those with more advanced knowledge can be given more challenging assignments.  Illinois Tech students will serve as teaching assistants (TAs) for the camp, which will provide them with the opportunity to reinforce their own learning by teaching the computational science that they have learned. 

Our goal is to host 20 high school students in the summer of 2019 and increase the number each year to 40 high school students steady state in the summer of 2021.  The summer camp is funded by tuition fees.  In order to broader access, we plan to offer tuition scholarships to those with financial need.  Due to the planned expansion, some grant funding may be required to support the tuition expenses in the short term.

\subsection{Strengthening undergraduate and graduate level computational science coursework} \label{Curr} 
Illinois Tech already has a substantial computational science curriculum.  However, it has several deficiencies, which we aim to address.

\subsubsection{\CurrExistName} \label{CurrExist} Illinois Tech quite a few several computational courses.  Computer science offers high performance computing at the undergraduate and graduate levels.  Applied mathematics BS through PhD students must all take at least one computational mathematics course and may choose from a handful of computational mathematics electives. There are two undergraduate computational physics courses, a computational chemistry course, and a computational biology course.

These courses teach Skills \ref{OneCPU}, \ref{EffOne}, and \ref{Natural}.  However, students tend to take only the courses in their own major.  Thus an applied mathematics student may not really have the chance to develop Skill \ref{Natural}.  The (co-)PIs, who teach some of these courses, will liaise  with other course instructors and the host departments to encourage non-majors to take these courses (Skill \ref{Other}).  This includes publicizing the relevant courses and exploring how we might lower the pre-requisite barriers.

\subsubsection{\RelSoftName} \label{RelSoft} In the Fall 2013 Choi and Hickernell piloted a one-credit course on Reliable Mathematical Software, primarily for their own graduate students.  The course covered truncation and round-off errors,
stopping criteria for adaptive software,
creating software collaboratively via Git, 
documentation,
input parsing and validation, and
reproducible computation.
Students completed a project consisting of a small software library or an extension to an existing library.  Several of the students in this course participated in developing GAIL) \cite{ChoEtal17b}.  This experience helped one student land an academic job teaching computer science, three students land computationally oriented positions in industry, and a fifth student gain admission to a PhD program.

In this project we will grow this to a standard three-credit undergraduate/graduate elective course.  In addition to a fuller treatment of the above topics, which address Skills \ref{Repro}--\ref{ContLib}, we will add material on working in a multi-lingual environment (Skill \ref{MultiLing}) and small-scale parallel computation (Skills \ref{TwoCPU} and \ref{MultiJobs}).  The prerequisite will be any undergraduate course that covers numerical computation.

\subsubsection{\LargeSCName} \label{LargeSC} Since 2006 the computer science department has offered a graduate course, Advanced Scientific Computation, which covers Skills \ref{Tight}, \ref{AnalVis}, \ref{MultiLing}, \ref{HighLib}, and \ref{EffLarge}.  Dr.\ Hong Zhang from Argonne National Laboratory has taught this course to students in computer science, mathematics, physics, and engineering. Students were exposed to cutting-edge analytic and algorithmic research projects and gained hands-on large-scale numerical programming experience. This course has been a launching point for many students into careers in computational science.  Several of the course projects were integrated into the Portable Extensible Toolobox for Scientific Computing (PETSc) \cite{petsc-web-page17}, benefiting the scientific community in large. At least eight students received summer internships at Argonne after taking her course, and three of her students eventually became post-doctoral researchers at Argonne.  

In the past, this course has run only every other year.  We plan to increase the frequency of this course to annually, serving a dozen or more students, by advertising its existence and benefits.  Students taking this course will be more competitive for the other opportunities below.

\subsubsection{\PhyChemBioCompSciName} \label{PhyChemBioCompSci} A new computational physics/chemistry/biology course using OpenScience Grid and XSEDE? \FJHNote{David and Jeff, something here for you?}

\subsection{\CODSummerName} \label{CODSummer}
Community colleges enroll approximately 45\% of the nation's undergraduate students, a disproportionate number of which are underrepresented minorities \cite{KnappKG12,nsfreport13}.  Nearly 90\% of these students have a goal of transferring to four-year institutions to complete their bachelor's degree, however the actual transfer rate is estimated to be only 25-40\% \cite{HoachlanderSH03,MelguizoKA11}.  Studies have shown that involving students, especially those from underrepresented groups, in  research activities decreases  their attrition rate and increases the probability they will pursue further education \cite{BarlowV04,JonesBV10}.

To  encourage the pursuit of bachelor's and advanced degrees by community college students, \JW has partnered in the past with the College of DuPage (COD) to recruit students from their associate degree programs  to engage in a ten-week summer internship program in his research group.  We propose to broaden this program to include several CISC associated research groups at Illinois Tech. Each year we will work with Prof.~Tom Carter of the physics department to advertise and recruit students to this program,  with a particular focus on underrepresented minorities.  These students will be matched with CISC labs based on their research interests and career goals.  Our goal is to aim for 

In the first phase of their internships, the cohort of COD students will spend a week in an intensive computing ``crash course'' that will include topics such as an introduction to linux, programming in python, and the use of high performance computing resources.  For the remaining nine weeks, CoD students will work directly in IIT labs on various computational projects, such as high performance Monte Carlo methods in Choi's and Hickernell's group, ??? in the Minh's group, ??? in Sun's group \JWNote{help me list a few}, and simulations of biomolecular complexes in the Wereszczynski group.  To foster discussions among the CoD students, weekly lunches will be organized.  In addition, at the end of the summer a research symposium will be organized in which students will present a short presentation on their work.  Following the summer period, we will track the career trajectories of these students with annual follow up surveys, and we will maintain mentoring relationship with these students as are appropriate. 

We will aim for four COD students in 2019, six in 2020, and six in 2021.  Students will be provided \$5000 stipends.  After the project ends, we hope to secure funding to continue from other sources, such as Research Experience for Undergraduate (REU) funding and REU add-on supplemental funding.


\subsection{\FellowName} \label{Fellow}
To provide more opportunities for the practice of the advanced CI skills that we are teaching our students, we plan to offer summer fellowships to Illinois Tech undergraduate and graduate students.  These will be awarded on a competitive basis and be provided to students who partake one of the following experiences
\begin{itemize}
\item Embedding themselves in a computational science research group outside their major, or
\item Joining a large-scale computational science research project at Argonne, Fermilab, or in a local company.
\end{itemize}

The former opportunity moves our students outside their silos.  The broadening experience will make them more valuable members of interdisciplinary teams performing large-scale computations.  It will help them learn and practice Skills \ref{Natural}--\ref{Other}.  

CISC began a series of lunchtime matchmaking seminars in the fall of 2017 to introduce computational scientists and engineers at Illinois Tech to each others' research.  CISC also hosted a seed grant competition and awarded one grant to a multidisciplinary team.  The goal of these ongoing activities is to promote more competitive proposals for external funding of interdisciplinary computational science research.  This goal will be aided by the introduction of CISC fellows, while at the same time our other efforts to raise the awareness of computational science research at Illinois Tech should promote quality applications for the CISC Fellowships.

The ongoing large-scale computational projects at Argonne and Fermilab will provide students the opportunity to learn and practice Skills \ref{TopMach}, \ref{AnalVis}, \ref{MultiLing}, \ref{Decision}, and \ref{Model}.  Argonne has several software development projects including PETSc \cite{petsc-web-page17} and xSDK \cite{XSDK17a}. Fermilab has a heterogeneous computing environment for high energy physics (HEPCloud) \cite{HEP18a} under continuous development.  Two of our external advisory board members, Lois Curfman McInnes (Argonne) and Burt Holzman (Fermilab) will aid us in placing CISC Fellows in these two national labs.

The (co-)PIs have had collaborations and contacts with Chicago-area companies involved in supporting or performing large-scale calculations as part of the research.  We will further develop these contacts and identify sources of possible fellowships.

Each year, starting in 2019, we will offer CISC fellowships to two undergraduates (\$5000 stipend each) and five graduate students (\$7000 stipend each).  The students will apply in the early part of the year, providing a synopsis of their proposed projects, their curriculum vitae, and letters of recommendation from their advisors and the project supervisors.  The management committee for this project---the (co-)PIs---will decide who will receive the fellowships.  

By the time that this grant is completed, we expect that our overall training program for computational scientists will have prepared a significant number of our to compete for funded internship opportunities in the national laboratories.  Moreover, we anticipate that embedding  students in research groups outside their silos will spark cross-disciplinary collaborations among research groups.  This in turn will lead to external funding of cross-disciplinary computational projects that will draw together and financially support computational science students from diverse majors.


\section{Broader Impacts}

\subsection{Educating the Next Computational Scientists} 
The standard science curricula has not kept pace with the potential of advanced CI for computational science.  Computer science courses teach the about the architectures and languages for large-scale computation, but they do not teach their students how to solve important scientific problems.  Computational mathematics, biology, chemistry, and physics courses do not educate their students in advanced CI.

The proposed computational science education program proposed here will set students on a path so that by the time they finish their graduate studies, they can be productive computational scientists working on interdisciplinary research teams discovering, predicting and optimizing new science.  Their work will lead to breakthroughs in our understanding the fundamental nature of matter, which will satisfy our curiosity.  Their work will lead to breakthroughs in health and medicine, and in manufacturing, which will improve the quality of our lives and the national economy. 

\subsection{Impacting Research in Computer Science, Mathematical Science, and Natural Science}
The interdisciplinary computational research which our proposed program encourages will feed interesting problems to the contributing disciplines:
\begin{itemize}
\item Complex quantitative models of natural science phenomena will challenge computational mathematicians and computer scientists to develop new algorithms requiring more advanced software and hardware to solve these models.  
\item The advances in CI will alter how computational mathematicians and computer scientists design algorithms and how they measure algorithmic efficiency.  The simple measures suited to single processor systems do not apply to advanced CI, and the replacement measures need to be developed.  Given the variety of architectures and languages, clock time is not necessarily a good choice.
\item Large-scale scientific computation enabled by advanced CI will answer existing questions in the natural sciences and prompt new research questions.  
\end{itemize}
The computational scientists that can answer these new problems will be those who are already proficient contributors and users of advanced CI, i.e., CICs and CIUs.  They must know there own discipline well enough to be able to contribute to answering solving these new problems.  But they must also be conversant in the other disciplines as well.

As an example, consider the situation of quasi-Monte Carlo (qMC) methods (Hickernell's research area), which are used for the evaluation of multi-dimensional integrals. From their 1960s until the early 1990s the developments in qMC were mainly theoretical.  Then in the mid-1990s, Paskov and Traub \cite{PasTra95} showed that qMC could price a collateralized mortgage obligation much more efficiently than simple or independent and identically distributed Monte Carlo (IIDMC).  This was surprising, since existing theory supporting showed that qMC was better than IIDMC up to dimension five or so, but Paskov and Traub's problem had dimension 360.  This computational result challenged mathematicians and computer scientists to re-visit the long-held theory, and the result was dozens of articles and a more complete understanding of the situations in which qMC is superior to MC, even for infinite dimensional integration.  New computation raised new research questions, which then were answered.

As another example, a couple of years ago, Jim\'enez Rugama, the former PhD student of Hickernell, was invited to join a research group of physicists at Fermilab to explore whether their Monte Carlo (MC) computations could be sped up.  The default MC algorithm being used was several decades old.  Jim\'enez Rugama was able to introduce the group to the much more efficient qMC methods.  In the course of implementing these methods, Jim\'enez Rugama discovered an anomaly in the calculations, which was traced to a bug in the long-standing physics code that generated the particle collision events.

\subsection{Modeling How Silos Can be Broken}
We recognize that computer science, mathematical science, and the several natural sciences are well-defined disciplines.  Someone existing only in one of these silos cannot take full advantage of advanced CI for scientific computation.  The solution is not build a new computational science silo.  The solution is to develop computational scientists that are deep in one discipline but break out of their own silo to be conversant in others.

As an analogy, the United States is comprised of people from a mixture of cultures.  Our strength comes not from homogenizing the cultural differences but from drawing on the strengths of our diverse cultural heritage while understanding the cultures that are not our own.  Most of the (co-)PIs have benefited by learning a language and culture other than the one they were born into.

This project will show other educational institutional how to break down the major discipline silos to foster high quality computational science.  The activities that we will implement will be models for others.  We will make our syllabi, training materials, and procedures available on repositories for others to follow and model.  We will publicize our successes and the lessons learned from failures in conference talks.  We will advise those interested to adopt our best practices.

\subsection{Codifying Good Computational Science Practice}
The practice of good computational science is built upon diverse ideas that are not necessarily contained in one text or reference book.  Some texts on computational science, such as \cite{TveEtal10a}, focus on numerical methods, while others focus, such as \cite{Thi13a}, focus on on scientific applications.  Neither of these types of texts give significant attention to parallel computing or software engineering for high performance computational science software libraries. Texts on high performance computing for computational scientists, such as \cite{MagEtal16a}, are incomplete in their coverage of numerical methods.

As we develop the new course described in Sect. \ref{RelSoft} and as we work with the CISC fellows, we will gather digestible resources describing good computational science practice

\subsection{Broadening Access to Computational Science Opportunities}

\section{Partnerships and External Advisory Board}
To maximize the impact of our project, we are partnering with Ms. April Welch, ??? and three outside organizations through key contacts
\begin{itemize}
\item Dr.\ Lois Curfman McInnes, Senior Computational Scientist, Argonne National Laboratory, Former PETSc lead, Present xSDK lead
\item Dr.\ Burt Holzman, 
\item 
\end{itemize}

\section{Evaluation}



\clearpage
\newpage


%\pagenumbering{arabic}
\setcounter{page}{1}

\bibliographystyle{spbasic}

{\renewcommand\addcontentsline[3]{} 
\renewcommand{\refname}{{\Large\textbf{References Cited}}}                   %%
\renewcommand{\bibliofont}{\normalsize}

\bibliography{FJH23,FJHown23,GregPapers,jeff}}

\end{document}

Review-specific criteria
1. Challenges addressed in training, education, and workforce development;
2. New modes of discovery and use of advanced CI resources, tools, and services in fundamental research enabled;
3. Advances in integrating skills in advanced CI as well as computational and data science and engineering into institutional and disciplinary curriculum/instructional material;
4. Steps to broaden access and community adoption with respect to the Nation’s scientific and engineering research workforce and advanced CI;
5. Stakeholders engaged and partnerships forged for collective impact;
6. Scalability to a large number of people directly and indirectly, and sustainability of key aspects beyond NSF funding; and
7. Plans for recruitment and assessment.

Proposals must clearly address the following solicitation-specific review criteria through well-identified proposal elements:
1. Are the training, education, and research workforce challenges identified sound?
2. What is the potential of the project to enable new modes of discovery and use of advanced CI resources, tools, and services in fundamental research?
3. How well would the project advance the goal of integrating skills in advanced CI as well as computational and data science and engineering into institutional and disciplinary curriculum/instructional material?
4. To what extent can the project meet its broadening access and community adoption challenges with respect to the Nation’s scientific and engineering research workforce and advanced CI?
5. How well would the project engage key stakeholders and forge partnerships for collective impact?
6. What is the potential for the project to scale and for its key aspects to be sustained beyond NSF funding?
7. Are the plans for recruitment and evaluation sound?
8. Are the plans for management and collaboration effective?

