%CyberTraining NSF Grant Feb 2018
\documentclass[11pt]{NSFamsart}
\usepackage[utf8]{inputenc}
\usepackage{xspace}
\usepackage[dvipsnames]{xcolor}
\usepackage[numbers]{natbib}
\usepackage{hyperref,accents,booktabs}
\usepackage{cleveref}
% This package prints the labels in the margin
%\usepackage[notref,notcite]{showkeys}




\thispagestyle{plain}
\pagestyle{plain}

\headsep-0.6in
\textwidth6.5in
\oddsidemargin0in
\evensidemargin0in
\textheight9in

\newcommand{\myshade}{85}
\colorlet{mylinkcolor}{violet}
\colorlet{mycitecolor}{Aquamarine}
\colorlet{myurlcolor}{YellowOrange}
\hypersetup{ %make the links stand out
	linkcolor  = mylinkcolor!\myshade!black,
	citecolor  = mycitecolor!\myshade!black,
	urlcolor   = myurlcolor!\myshade!black,
	colorlinks = true,
}

\providecommand{\FJHickernell}{Hickernell}

% Everyone feel free to add their own note definitions here
\newcommand{\FJHNote}[1]{{\textcolor{blue}{FJH: #1}}}






\begin{document}
\setlength{\leftmargini}{2.5ex}

\centerline{\textbf{\Large Management and Coordination Plan}}

\bigskip

\subsection*{Management Team} This project will be managed by the (co-)PIs, who will meet every one to two months to report on the progress of the project and discuss challenges that arise.  The Management Team together with the senior personnel and Ms.~April Welch will oversee the following initiatives

\begin{tabular}
{>{\flushleft}p{0.6\textwidth}>{\flushleft}p{0.4\textwidth}}
\textbf{Task} & \textbf{Responsible Party}
\ref{}
\end{tabular}

\subsection*{External Advisors}  An advisory board consisting of the following members will facilitate partnerships their organizations and advise on help assess how well we are meeting our goals.  These external advisors will be
\begin{itemize}
\item Lois Curfman McInnes, Senior Computational Scientist, Argonne National Laboratory,
\item Burt Holzman, Assistant Division Director, Fermilab
\item Norm Lederman, Distinguished Professor, Mathematics and Science Education, Illinois Tech

\end{itemize}

PI meeting at the NSF

\subsection*{Timeline}

\end{document}

1. Management and Coordination Plan (2 pages): Each proposal must contain a clearly-labeled Management and Coordination Plan that includes: 1) the specific roles of the PI, co-PIs, other Senior Personnel and paid consultants at all institutions involved; 2) how the project will be managed across institutions and disciplines; 3) identification of the specific coordination mechanisms; and 4) pointers to the budget line items that support these management and coordination mechanisms.
